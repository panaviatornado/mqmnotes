\documentclass[12pt]{article}
\usepackage{titling}
\usepackage{amssymb}
\usepackage{amsmath}
\usepackage{extpfeil}
\usepackage{marvosym}
\usepackage{graphicx}
\usepackage{setspace}
\usepackage{tabto}
\usepackage{wasysym}
\usepackage{textcomp}
\usepackage{braket}
\usepackage{amsthm}
\usepackage{thmtools}
\usepackage[xcolor = RGB, framemethod = TikZ]{mdframed}
%\usepackage[dvipsnames]{xcolor}
\usepackage{subfigure}
\usepackage{mathtools}
\usepackage{ragged2e}
%\usepackage{hhline}
\usepackage{float}
%\usepackage{relsize}
\usepackage{mathrsfs}
\usepackage[margin=1in]{geometry}
\usepackage{units}
%\usepackage{bbm}
%\usepackage{MnSymbol}
\usepackage{varwidth}
\usepackage{titlesec}
%\usepackage[round,colon,authoryear]{natbib}
%\usepackage{nth}
%\usepackage{hyperref}
%\usepackage[russian]{babel}
%\usepackage[utf8]{inputenc}
%\usepackage[T2A,T1]{fontenc}
%\usepackage[latin, british]{babel}

\usepackage{hyperref}
\usepackage{mleftright}
\usepackage{comment}
\usepackage{tensor}
\usepackage{siunitx}


\usepackage{bm}


\numberwithin{equation}{section}

%\titleformat{\subsection}[runin]{\normalfont\large\bfseries}{\thesubsubsection}{1em}{}

\setlength\parindent{0pt}

%\theoremstyle{definition}
%\newtheorem{prop}{Proposition}
%\newtheorem{cor}{Corollary}
%\newtheorem{lemma}{Lemma}


%%%%%%%%%%%%
%%%Colors%%%
%%%%%%%%%%%%

\colorlet{thmline}{red}
\definecolor{thmbg}{rgb}{1,0.9,0.9}
\definecolor{propline}{rgb}{1,0.27,0}
\definecolor{propbg}{rgb}{1,0.854,0.752}
\definecolor{corline}{rgb}{0.812,0.325,0}
\definecolor{corbg}{rgb}{1,0.854,0.752}
\colorlet{lemline}{orange}
\definecolor{lembg}{rgb}{1,0.824,0.702}
\definecolor{exampleline}{rgb}{0,0.898,0.565}
\colorlet{examplebg}{white}
\colorlet{defnline}{blue}
\definecolor{defnbg}{rgb}{0.9,0.9,1}
\definecolor{remline}{rgb}{1,0.843,0}
\colorlet{rembg}{white}

%%%%%%%%%%%%%%%%%%%%
%%%Theorem Styles%%%
%%%%%%%%%%%%%%%%%%%%



\mdfdefinestyle{theoremstyle}{
	middlelinewidth = 3,
	roundcorner  = 10 pt,
	leftmargin = -12,
	rightmargin = -12,
	backgroundcolor  = thmbg,
	middlelinecolor = thmline,
	innertopmargin = -2pt ,
	splittopskip = \topskip ,
	ntheorem = true,
	}

\mdfdefinestyle{propstyle}{
	middlelinewidth = 3,
	roundcorner  = 10 pt,
	leftmargin = -12,
	rightmargin = -12,
	backgroundcolor  = propbg,
	middlelinecolor = propline,
	innertopmargin = -2pt,
	splittopskip = \topskip ,
	ntheorem = true,
	}

\mdfdefinestyle{corstyle}{
	middlelinewidth = 3,
	roundcorner  = 10 pt,
	leftmargin = -12,
	rightmargin = -12,
	backgroundcolor  = corbg,
	middlelinecolor = corline,
	innertopmargin = -2pt ,
	%splittopskip = \topskip ,
	ntheorem = true,
	}

\mdfdefinestyle{lemstyle}{
	middlelinewidth = 3,
	roundcorner  = 10 pt,
	leftmargin = -12,
	rightmargin = -12,
	backgroundcolor  = lembg,
	middlelinecolor = lemline,
	innertopmargin = -2pt ,
	splittopskip = \topskip ,
	ntheorem = true,
	}

\mdfdefinestyle{examplestyle}{
	middlelinewidth = 3,
	roundcorner  = 10 pt,
	leftmargin = -12,
	rightmargin = -12,
	backgroundcolor  = examplebg,
	middlelinecolor = exampleline,
	%innertopmargin = 0pt ,
	splittopskip = \topskip ,
	ntheorem = true,
	}

\mdfdefinestyle{defnstyle}{
	middlelinewidth = 3,
	roundcorner  = 10 pt,
	leftmargin = -12,
	rightmargin = -12,
	backgroundcolor  = defnbg,
	middlelinecolor = defnline,
	%innertopmargin = 0pt ,
	splittopskip = \topskip ,
	ntheorem = true,
	}

\mdfdefinestyle{remstyle}{
	middlelinewidth = 3,
	roundcorner  = 10 pt,
	leftmargin = -12,
	rightmargin = -12,
	backgroundcolor  = rembg,
	middlelinecolor = remline,
	%innertopmargin = 0pt ,
	splittopskip = \topskip ,
	ntheorem = true,
	}


%%%%%%%%%%%%%%%%%%%%%%%%%%
%%%Theorem Environments%%%
%%%%%%%%%%%%%%%%%%%%%%%%%%




\theoremstyle{plain}
\newtheorem{thnum}{Theorem}[section]



\declaretheoremstyle[bodyfont = \normalfont,
]{example}




\declaretheorem[sharenumber = thnum, qed = $\square$, 
preheadhook = {\begin{mdframed}[style = theoremstyle]},
postfoothook = \end{mdframed}
]{theorem}


\declaretheorem[sharenumber = thnum, name = Proposition, qed = $\square$,
preheadhook = {\begin{mdframed}[style = propstyle]},
postfoothook = \end{mdframed}
]{prop}

\declaretheorem[sharenumber = thnum, name = Corollary, qed = $\square$,
preheadhook = {\begin{mdframed}[style = corstyle]},
postfoothook = \end{mdframed}
]{cor}

\declaretheorem[sharenumber = thnum, name = Lemma, qed = $\square$,
preheadhook = {\begin{mdframed}[style = lemstyle]},
postfoothook = \end{mdframed}
]{lemma}

\declaretheorem[style = example, sharenumber = thnum,
preheadhook = {\begin{mdframed}[style = examplestyle]},
postfoothook = \end{mdframed}
]{example}

\declaretheorem[style = example,sharenumber = thnum, name = Definition, qed = $\square$,
preheadhook = {\begin{mdframed}[style = defnstyle]},
postfoothook = \end{mdframed}
]{defn}

\declaretheorem[style = example, sharenumber = thnum, qed = $\square$,
preheadhook = {\begin{mdframed}[style = remstyle]},
postfoothook = \end{mdframed}
]{remark}


%%%%%%%%%%%%%%%%%%%%%%%%%%%%%%%%%%%%%%%%%%%%%%%%%%%%%%%%%%%%%

\declaretheorem[numbered = no, qed = $\square$, name = Theorem,
preheadhook = {\begin{mdframed}[style = theoremstyle]},
postfoothook = \end{mdframed}
]{theorem*}


\declaretheorem[numbered = no, name = Proposition, qed = $\square$,
preheadhook = {\begin{mdframed}[style = propstyle]},
postfoothook = \end{mdframed}
]{prop*}

\declaretheorem[numbered = no, name = Corollary, qed = $\square$,
preheadhook = {\begin{mdframed}[style = corstyle]},
postfoothook = \end{mdframed}
]{cor*}

\declaretheorem[numbered = no, name = Lemma, qed = $\square$,
preheadhook = {\begin{mdframed}[style = lemstyle]},
postfoothook = \end{mdframed}
]{lemma*}

\declaretheorem[style = example, numbered = no, name = Example,
preheadhook = {\begin{mdframed}[style = examplestyle]},
postfoothook = \end{mdframed}
]{example*}

\declaretheorem[style = example, numbered = no, name = Definition, qed = $\square$, name = Definition,
preheadhook = {\begin{mdframed}[style = defnstyle]},
postfoothook = \end{mdframed}
]{defn*}

\declaretheorem[style = example, numbered = no, qed = $\square$, name = Remark,
preheadhook = {\begin{mdframed}[style = remstyle]},
postfoothook = \end{mdframed}
]{remark*}


%Weak Convergence Arrow
\newextarrow{\xrightharpoon}{0529}{\relbar\relbar\rightharpoonup}





\begin{comment}


\theoremstyle{plain}
\newtheorem{thnum}{Theorem}[section]

\declaretheoremstyle[bodyfont = \normalfont,
%postheadspace=0.5em,
%mdframed={
%		backgroundcolor=white, hidealllines=true,
%		skipabove=0pt, skipbelow=0pt,
%		innertopmargin=2pt, innerbottommargin=2pt, innerleftmargin=0pt, innerrightmargin=0pt,
%		}
]{example}
\declaretheorem[sharenumber = thnum, qed = $\square$,shaded={rulecolor=Red,
rulewidth=2pt, bgcolor={rgb}{1,0.9,0.9}, margin = 10pt}]{theorem}
\declaretheorem[sharenumber = thnum, name = Proposition, qed = $\square$,shaded={rulecolor=RedOrange,
rulewidth=2pt, bgcolor={rgb}{1,0.854,0.752}, margin = 10pt}]{prop}
\declaretheorem[sharenumber = thnum, name = Corollary, qed = $\square$,shaded={rulecolor=BurntOrange,
rulewidth=2pt, bgcolor={rgb}{1,0.854,0.752}, margin = 10pt}]{cor}
\declaretheorem[sharenumber = thnum, name = Lemma, qed = $\square$,shaded={rulecolor=Orange,
rulewidth=2pt, bgcolor={rgb}{1,0.824,0.702}, margin = 10pt}]{lemma}
\declaretheorem[style = example, sharenumber = thnum,shaded={rulecolor=SpringGreen,
rulewidth=2pt, bgcolor={rgb}{1,1,1}, margin = 10pt}]{example}
\declaretheorem[style = example,sharenumber = thnum, name = Definition, qed = $\square$,shaded={rulecolor=Blue,
rulewidth=2pt, bgcolor={rgb}{0.9,0.9,1}, margin = 10pt}]{defn}
\declaretheorem[style = example, sharenumber = thnum, qed = $\square$,shaded={rulecolor=Goldenrod,
rulewidth=2pt, bgcolor={rgb}{1,1,1}, margin = 10pt}]{remark}

\declaretheorem[numbered = no, qed = $\square$, name = Theorem]{theorem*}
\declaretheorem[numbered = no, name = Proposition, qed = $\square$,shaded={rulecolor=RedOrange,
rulewidth=2pt, bgcolor={rgb}{1,0.854,0.752}, margin = 10pt}]{prop*}
\declaretheorem[numbered = no, name = Corollary, qed = $\square$,shaded={rulecolor=BurntOrange,
rulewidth=2pt, bgcolor={rgb}{1,0.854,0.752}, margin = 10pt}]{cor*}
\declaretheorem[numbered = no, name = Lemma, qed = $\square$,shaded={rulecolor=Orange,
rulewidth=2pt, bgcolor={rgb}{1,0.824,0.702}, margin = 10pt}]{lemma*}
\declaretheoremstyle[numbered = no, bodyfont = \normalfont]{example*}
\declaretheorem[style = example*, name = Example,qed = $\square$,shaded={rulecolor=lime,
rulewidth=2pt, bgcolor={rgb}{1,1,1}, margin = 10pt}]{example*}
\declaretheorem[style = example*, name = Definition, qed = $\square$,shaded={rulecolor=Blue,
rulewidth=2pt, bgcolor={rgb}{0.9,0.9,1}, margin = 10pt}]{defn*}
\declaretheorem[style = example*, name = Remark, qed = $\square$,shaded={rulecolor=Goldenrod,
rulewidth=2pt, bgcolor={rgb}{1,1,1}, margin = 10pt}]{remark*}

\end{comment}




\newcommand{\nin}{\notin}

\renewcommand{\baselinestretch}{1.3}

\renewcommand{\d}{\mathrm{d}\hspace{-0.025cm}}
\renewcommand\qedsymbol{\textit{q.e.d.}}

\renewcommand{\epsilon}{\varepsilon}
\renewcommand{\phi}{\varphi}
\renewcommand{\theta}{\vartheta}
\renewcommand{\cong}{\simeq}

\DeclareMathOperator*{\esssupp}{ess \, supp}
\DeclareMathOperator*{\supp}{supp}
\DeclareMathOperator{\intr}{int}
\DeclareMathOperator*{\esssup}{ess \, sup}
\DeclareMathOperator{\sgn}{sgn}
\DeclareMathOperator{\rnk}{rnk}
\DeclareMathOperator{\loc}{loc}
\DeclareMathOperator{\vol}{vol}
\DeclareMathOperator{\img}{img}
\DeclareMathOperator{\ran}{ran}
\DeclareMathOperator{\euc}{Eucl}
\DeclareMathOperator{\spn}{span}
\DeclareMathOperator{\id}{Id}
\DeclareMathOperator{\dist}{dist}
\DeclareMathOperator{\Tr}{Tr}
\DeclareMathOperator{\pr}{pr}
\DeclareMathOperator{\spec}{spec}



\let\Re\relax
\DeclareMathOperator{\Re}{\frak R}

\newcommand{\bitext}[1]{\textit{\textbf{#1}}}

\newcommand{\subtitle}[1]{%
  \posttitle{%
    \par\end{center}
    \begin{center}\large#1\end{center}
    \vskip0.5em}%
}

%\renewcommand{\proofname}{\textit{Proof.}}


\makeindex

\begin{document}
\title{Mathematical Quantum Mechanics\\
Prof. Heinz Siedentop \& Dr. Robert Helling}
\author{Martin Peev}
\subtitle{Unofficial Lecture Notes}
\date{}
\maketitle

\newpage

\tableofcontents

\newpage

%\setcounter{section}{-1}

\section{Review of Quantum Mechanics}

A state is a non-zero $\psi \in \mathcal H$ where $\mathcal H$ is a Hilbert space. Actually it is an equivalence class with 
\[
	\psi_1 \sim \psi_2 \iff \exists c \in \mathbb{C}^*: c \psi_1 = \psi_2,
\]
so actually it is a ray in $\mathcal H$.

Further the Hilbert space is a $\mathbb{C}$-vector space with a positive definite hermitean inner product $\Braket{\cdot, \cdot}: \mathcal H^2 \rightarrow \mathbb{C}$, that is linear in the second slot, which induces a norm $\|\psi\| = \sqrt{\Braket{\psi,\psi}}$, which makes $\mathcal H$ into a complete normed space.

At least as important are operators or linear maps $A: \mathcal H \rightarrow \mathcal H$.



Stern-Gerlach experiment [there was a drawing]: The Hamiltonian of the interaction with the
magnetic field is given by 
\[
	H_{\text{int}} = \bm S \cdot \bm B(x)
\]
$B$ is mostly in $z$-direction hence we get
\[
	\bm B \cdot \bm S \approx \bm B(x) S_z, 
\]
with
\[
	S_z = \frac{1}{2} \begin{pmatrix}
	1 & 0\\
	0 & -1
	\end{pmatrix}
\]
and the force being proportional to the gradient of the potential, i.e.
\[
	F \sim -\bm S \cdot \nabla B,
\]
hence this experiment discriminates the two eigenspaces of $H_{\text{int}}$.

The two roles of the Hamiltonian are thus
\begin{itemize}
	\item giving the time evolution,
	\item measuring (splitting the eigenspaces).
\end{itemize}

$S_z$ is a $2\times 2$ complex matrix, operator on the Hilbert space $\mathbb{C}^2$. Everything is simple
\begin{itemize}
	\item it is hermitean $S_z^\dagger = S_z$
	\item discrete eigenvalues, solve $S_z \psi = s_z \psi$, with $s_z \in \mathbb{C}$.
	\item $S_z = \frac{1}{2} \ket{\uparrow}\bra{\uparrow} +
          \mleft(- \frac{1}{2} \mright) \ket{\downarrow}
          \bra{\downarrow}$ (weighted sum of eigenprojectors, spectral
          theorem)
	\item defined on all of $\mathbb{C}^2$
	\item norm: maximal $|\hbox{eigenvalue}|$.
\end{itemize}

(Mathematical) problems arise when $\dim \mathcal H = \infty$. In that situation the same points (but in different order )


\underline{By example:}
\begin{enumerate}
	\item[1)] Eigenvalue equation $A\psi = a \psi$ can have no non-trivial solution in $\mathcal H$, e.g.
	\[
		- \Delta \psi = E \psi \qquad \text{in  } \mathcal H = L^2 \mleft( \mathbb{R}^n \mright).	
	\]
	is only solved by $\psi = 0$ in $\mathcal H$.
	
	"Plane wave solutions":
	\[
		\psi(x) = e^{i k x}	
	\]
	with $\|k\|^2 = E$ are not square integrable, this problem is associated with the continuous spectrum.
	
	Way out: Instead of $A \psi = \lambda \psi$ write $(A - \lambda \id)\psi = 0 $
	or $A-\lambda \id $ is not invertible, with the last assertion
        being the more general than the previous in the case of
        infinite dimensions. Hence we shall define
	\[
		\lambda \in \spec(A) :\iff A- \lambda \id \text{ is not invertible}.
	\]
	
	\item[2)] The norm of $A$ is defined as 
	\[
		\|A\|:= \sup_{\psi \in \mathcal H \setminus \{0\}}	\frac{\|A\psi\|}{\|\psi\|} = \sup_{\|\psi\| = 1} = \| A \psi\| \in [0,\infty]
	\]
	If $\|A\| = \infty$, $A$ is called \underline{unbounded}. These cause headaches
	\begin{enumerate}
	
	
	\item[a)] $A$ is not continuous in this case, since
	\[
		A \text{ continuous} \iff \|A\|< \infty	
	\]
	
	Fact of life: $[q,p] = i \implies$ at least one of $p,q$ is unbounded.
	
	\item[b)] In fact $A$ cannot be defined on all of $\mathcal H$, e.g. for $\mathcal H = L^2(\mathbb{R})$, the position operator is given by 
	\[
		(x\psi)(x) = x\psi(x)	
	\]
	can leave $\mathcal H$. For example $\psi(x) = \mleft(1+|x| \mright)^{\nicefrac{2}{3}} \in \mathcal H$, but $x \psi(x) = x \mleft(1+|x| \mright)^{\nicefrac{2}{3}} \notin \mathcal H$.
	
	Hence we have to worry about the domain of definition $\mathcal D(A)$ of an unbounded operator $A$.
	Usually, we want $\mathcal D(A)$ to be dense in $\mathcal H$.
	
	
	\end{enumerate}
	
	\item[3)] The adjoint operator: A nice poperty of Hilbert spaces is that via the scalar product, they are their own duals, i.e. every linear map $f: \mathcal H \rightarrow \mathbb{C}$ defines a $\phi \in \mathcal H$ such that 
	\[
		\forall \psi \in \mathcal H : f(\psi) = \Braket{\phi,\psi}	
	\]
	
	Now take $\phi \in \mathcal H$, such that 
	\[
		f_{\phi}: \begin{aligned} \mathcal D(A) &\longrightarrow \mathbb{C}	\\
			\psi &\longmapsto \Braket{\phi,A\psi}
		\end{aligned}
	\]
	extends to all $\mathcal H$ or equivalently is bounded.
	
	Then there is $\eta_{\phi} \in \mathcal H$ such that $f_{\phi}(\psi) = \Braket{\eta_{\phi},\psi}$.
	
	$\eta_{\phi}$ is usually denoted as ``$A^{\dagger}\phi$". Then 
	\[
		A^{\dagger} \colon\phi \longmapsto \eta_{\phi} = A^{\dagger}\phi	
	\]
	is linear and \[
	\mathcal D\mleft( A^{\dagger} \mright) := \mleft\{ \phi \in \mathcal H \, \big| \, f_{\phi} \text{ is bounded} \mright\}
	\]
	(i.e. you are not free to choose!)
	
	In Quantum mechanics on learns that \underline{self-adjoint} operators have a real spectrum and are potential observables. However, now we have to worry whether $\mathcal D(A) \overset{?}{=} \mathcal D\mleft(A^{\dagger}\mright)$.
	
\end{enumerate}

\begin{defn}
	A densely defined operator $A$ is called \underline{symmetric} iff 
	\[
		\forall \phi, \psi \in \mathcal D(A): \; \Braket{\phi, A \psi} = \Braket{A\phi,\psi}	
	\]
	Note that this only says that $A^{\dagger}\big|_{\mathcal D(A)} = A$ (that $\mathcal D(A) \subset \mathcal D \mleft( A^{\dagger} \mright)$ is already implied by the definition). However, it could be that $\mathcal D \mleft( A^{\dagger} \mright) \supsetneq \mathcal D \mleft( A \mright)$.
	
	A symmetric operator $A$ is called \underline{self-adjoint} iff 
	\[
		\mathcal D\mleft(A^{\dagger}\mright) = \mathcal D(A).	
	\]	
	
\end{defn}

\begin{example*}
       
        N.B. there is an extended write-up of this example in the file {\tt pextension.pdf}. 

	Let $\mathcal H = L^2 \big( [0,1]\big)$ and consider the
        operator $(p\psi)(x) = i \overline{\psi'(x)}$ for \[
	\mathcal D(p) \subset \mathscr C_a := \mleft\{ f \in \mathcal H \, \big| \, f \text{ absolutely continuous with derivative in } \mathcal H \mright\}.	
	\] Is this operator symmetric?	
	\[
		\Braket{\phi,p\psi}=\int\limits_0^1 \bar\phi(x) i \frac{d}{d x} \phi(x) \d x = \underbrace{ \int\limits_0^1 \overline{i \frac{d}{d x} \phi(x)} \phi(x) \d x	}_{ = \Braket{p\phi,\psi}} + i \bar \phi(x) \psi(x) \big|_{0}^1
	\]
	Since the last term has to equal 0, for $p$ to be symmetric, we have to impose the condition $\psi(0) = \psi(1) = 0$ so actually 
	\[
		\mathcal D(p) = \mleft\{ \psi \in \mathscr C_{a} \, \big| \, \psi|_{\{0,1\}} = 0 \mright\} 	
	\]
	Imposing these boundary conditions on $\psi$ makes it unnecessary for $\phi$, hence $\mathcal D\mleft(A^{\dagger}\mright) \supsetneq \mathcal D(A)$.
	
	One eigenvalue of the operator $p^{\dagger}$ is $\pm i$, since
	\[
		i \frac{d}{d x} \phi = \pm i \phi	\implies \phi = e^{\pm i x} \in \mathscr C_a.
	\]
	In particular we get that $\mathcal D \mleft(p^*\mright) = \mathscr C_a$.
	
	However, this means that $p^* \psi_{\pm} = \pm i \psi_{\pm}$	which means that $\Braket{\psi,p^*\psi}$ is not necessarily real, i.e. $p^*$ is not an observable. 
	
	We observe that $\dim \mathcal{D}(p^*)/\mathcal{D}(p) = 2$.
	
	$\mathcal D_{\pm} = \mleft\{ \text{eigenspace of eigenvalue }\pm i \mright\}$
	
\end{example*}

To solve this problem a possible strategy is to enlarge $\mathcal D(p)$ by a one-dimensional subspace of $\mathcal{D}(p^*)$, $\mathcal D(p_e)$ such that $\mathcal D(p_e^*) = \mathcal D (p_e)$.

We have to find to find a good one dimensional subspace $\mathcal D_a$ of $\mathcal D_+ \oplus \mathcal{D}_-$. Hence let 
\[
	\psi \in \mathcal D_a \; \qquad \psi = \tilde \psi_+ + \tilde \psi_i, \quad \tilde\psi_{\pm} \in \mathcal D_{\pm}.
\]
This decomposition is unique and defines a map $S: \mathcal D_+ \rightarrow \mathcal{D}_-$ such that
\[
	\tilde\psi_+ \in\mathcal{D}_+ \implies \tilde\psi_+ + S\mleft( \tilde\psi_+ \mright) \in \mathcal D_a.
\]

To see whether the new extended $p_e$ is symmetric for this
\[
	0 = \Braket{p^* \mleft(\tilde\psi_+ + S \tilde\psi_+  \mright) , \tilde\psi_+ + S\tilde\psi_+ } - \Braket{ \tilde\psi_+ + S\tilde\psi_+ , p^* \tilde\psi_+ + S\tilde\psi_+} = \cdots = -2i \mleft( \Braket{\tilde\psi_+,\tilde\psi_+} - \Braket{S\tilde\psi_+,S\tilde\psi_+} \mright),
\]
	where we used $p^* \psi_+ = i \psi_+$.
	
	Hence for $p_e$ to be symmetric we need $S$ to be unitary. It turns out that this condition is also sufficient.


In the general case we also have 
\[
	\mathcal D \mleft( A \mright) \oplus \mathcal D_+ \oplus \mathcal D_- = \mathcal{D} \mleft(A^+ \mright).
\]
We have the freedom to choose a self-adjoint extension for each $S: \mathcal D_+ \rightarrow \mathcal D_-$ unitary. Then 
\[
	\mathcal D_a = \mleft\{ \psi_+ + S \psi_+ \, \big| \, \psi \in \mathcal D_+ \mright)
\]

In our case, $\dim \mathcal D_{\pm} = 1$, so the possible unitary operators are $S_{\alpha}\psi_+ = e^{i \alpha} \psi_-$, for $\alpha \in [0,2\pi)$. Thus for $\psi \in \mathcal D(q) + \mathcal D_a$ we have
\[
	\psi(0) = e^{i \beta} \psi(1).
\]
I.e. $p$ is self-adjoint on the quasi-periodic functions, and thus the
spectrum of an extension $p_{\alpha}$, for a given choice of $\alpha$,
is $2\pi \mathbb Z + \alpha$. As the eigenvalues are the possible
outcomes of the measurement, we see that the choice of the
self-adjoint extension has an observable consequence.

\begin{theorem*}[Stone's Theorem]
		For every self-adjoint operator $A$ there exists a unitary representation $U: \mathbb{R} \rightarrow U (\mathcal H), \; U (t) U(s) = U(t+s) $ such that 
		\[
			\forall \psi \in \mathcal D(A): \frac{d}{d t} U(t) \psi \Big|_{t=0} = i A \psi		
		\]
\end{theorem*}

If we would not act on quasi-periodic functions, the unitary
representation acting as translations would shift the wave function
out of the intergral and thus would not be invertible and in
particular not unitary.

\newpage

\section{One-Particle Operators}


The standard quantum-mechanical ``self-adjoint" operator is $H_0 = - \Delta$. For $\psi \in \mathscr C_0^{\infty} \mleft( \mathbb{R}^3 \mright)$ we have
\[
	\Braket{\phi,-\Delta \psi} = \sum_{j} \int\limits_{\mathbb{R}^3} \bar\phi(x) \mleft( - \Delta \psi \mright)(x) \d x = \sum_j \int \overline{ \nabla \phi}(x) \cdot \nabla \psi(x) \d x = \sum_j \int \overline{- \Delta \phi}(x)\psi(x) = \Braket{-\Delta \phi , \psi}
\]

Since $\mathscr C_0^{\infty} \mleft( \mathbb{R}^3 \mright)$ is dense
in $L^2 \mleft( \mathbb{R}^3 \mright)$ this is symmetric.

Other examples of operators are 
\begin{align*}
	H_z &= - \Delta - \frac{z}{\|x\|}\\
	H & = - \Delta + x^2\\
	H& = - \Delta - \alpha \chi_M
\end{align*}

\begin{defn}[Fourier Transform]
	The Fourier transform of $\psi \in L^1\mleft( \mathbb R^n \mright) $ is given by 
	\[
		\mleft(\mathcal F \psi \mright)(\xi) = \frac{1}{\mleft(2\pi \mright)^{\nicefrac{n}{2}}} \int\limits_{\mathbb{R}^n} e^{- i \xi \cdot x} \psi(x) \d x 
	\]
\end{defn}

\begin{defn}[Japanese Symbol]
	\[
		\Braket{x} = \sqrt{1+\|x\|^2}	
	\]
\end{defn}

For $\psi(x):= \frac{1}{\Braket{x}^{3-\alpha}}$ it holds that $\psi \notin L^1$ but $\psi \in L^2$, for $\alpha$ small.

\begin{defn}
	Let us consider the semi-norms
	\[
		\|\psi\|_{\alpha, \beta} := \sup_{x \in \mathbb{R}^n} \mleft| x^\alpha \mleft( \partial^\beta \psi \mright)(x) \mright|,	
	\]
	with $\alpha, \beta \in \mathbb{N}_0^{n}$, where 
	\[
		x^\alpha= x^{\alpha_1}_1 \cdots x^{\alpha_n}_n, \quad \partial^\beta := \partial^{\alpha_1}_1	\cdots \partial^{\alpha_n}_n.
	\]
	Then the Schwartz space on $\mathbb{R}^n$ is defined as
	\[
		\mathscr S \mleft( \mathbb{R}^n \mright) := \mleft\{
                \psi \in \mathbb{C}^{\infty} \mleft( \mathbb{R}^n
                \mright) \, \big|\, \forall \alpha,\beta\colon \|\psi\|_{\alpha, \beta} < \infty \mright\},
	\]
	with the complete metric 
	\[
		d(\psi,\phi) := \sum_{\alpha, \beta \in \mathbb{N}^n} 2^{-|\alpha|-|\beta|}	\frac{\|\psi-\phi\|_{\alpha,\beta}}{1+\|\psi-\phi\|_{\alpha,\beta}}.
	\]
\end{defn}
\begin{defn}[Tempered Distributions]
	The dual space of $\mathscr{S}$, 
	\[
		\mathscr{S}' := \mleft\{ T : \mathscr{S}\mleft(\mathbb{R}^n \mright) \rightarrow \mathbb{C} \, \big| \, \text{linear, continuous} \mright\}	
	\]
	is called the space of tempered distributions.
\end{defn}

We shall define the Fourier transform on $\mathcal S'$ by 
\[
	\Braket{\mathcal FL , \psi} := \Braket{L, \mathcal F \psi},
\]
where $L \in \mathscr S', \psi \in \mathscr S$ and $\Braket{L,\psi}:=
L(\psi)$. This extens what holds for regular distributions
corresponding to $L^1$-functions.

The delta function $\delta_x : f \mapsto f(x)$ is a continuous functional on $\mathcal S$ since
\[
	\forall \psi \in S: \mleft| \delta_x f \mright| = |f(x)| \leqslant \|f\|_{0,0}.
\]
The Fourier transform of the delta function is given by
\[
	\Braket{\mathcal F \delta , \phi} = \Braket{\delta, \hat \phi} = \hat \phi(0) = \frac{1}{\mleft(2\pi \mright)^{\nicefrac{n}{2}}} \int e^{i 0 x} \phi(x) \d x = \Braket{\frac{1}{(2\pi)^{\nicefrac{n}{2}}}, \phi},
\]
i.e. $\mathcal F \delta = \frac{1}{(2\pi)^{\nicefrac{n}{2}}}$.


If $g$ is measurable and decays faster than any power, i.e. 
\[
	\exists C \exists M \forall x \in \mathbb{R}^n \mleft| \Braket{x}^{-M} g(x) \mright| < C
\]
then it gives rise to a {\sl regular} distribution, i.e. let $L_g$ be the functional defined by
\[
	\Braket{L_g, \phi}:= \int g(x) \phi(x) \d x
\]
for all $\phi \in \mathscr S\mleft(\mathbb{R}^n \mright)$.
One has to check that 
\begin{enumerate}
	\item[1)] $L_g$ is a continuous functional.
	\item[2)] $L_g$ with $g \in L^2 \mleft( \mathbb{R}^n \mright)$ is a tempered distribution.
\end{enumerate}


\begin{defn}[Sobolev space]
	A Sobolev space 
	\[
		H^{\alpha} \mleft( \mathbb{R}^d \mright) = \mleft\{	\psi \, \big| \, \mleft\| \Braket{\xi}^{\alpha} \hat\psi \mright\|_{L^2} < \infty \mright\}.
	\]
	Further we define 
	\begin{align*}
		H^1 \mleft( \mathbb{R}^n \mright): \Braket{\phi,\psi}_{H_1} := \int \Braket{\xi}^2 \overline{\hat\phi}(\xi) \overline{\psi}(\xi) \d \xi \\
		H^2 \mleft( \mathbb{R}^n \mright): \Braket{\phi,\psi}_{H_2} := \int \Braket{\xi}^4 \overline{\hat\phi}(\xi) \hat\psi(\xi) \d \xi
	\end{align*}
\end{defn}

On this space we can define the Hamiltonian 
\[
	-\Delta : \begin{aligned}
		H^2 \mleft( \mathbb{R}^n \mright) \subset L^2 \mleft( \mathbb{R}^n \mright) &\longrightarrow L^2 \mleft(\mathbb{R}^n \mright)\\
		\psi & \longmapsto \mathcal F^{-1} \mleft( |\cdot|^2 \hat\psi \mright)
	\end{aligned}
\]

Check that this is self-adjoint under the assumption that $F$ is unitary.
%\[
%	-\Delta: \mathcal{F}^-1 |\xi|^2 \mathcal F
%\]

\begin{defn}[Positivity]
	We say that $A$ is positive $A \geqslant 0$ iff
	\[
		\forall \psi \in \mathcal D(A) : \Braket{\psi, A \psi} \geqslant 0.	
	\] 
	In particular a positive operator is always symmetric.
\end{defn}


\begin{defn}
	Suppose that $A$ is self-adjoint in $\mathcal G$. Let 
	\[
		\mathcal Q(A) := \mathcal D \mleft( |A|^{\nicefrac{1}{2}} \mright),	
	\]
	be the quadratic form domain.
\end{defn}

\begin{example*}
	\begin{align*}
		\mathcal Q (-\Delta) &= \mathcal D \mleft( \mleft(-\Delta \mright)^{\nicefrac{1}{2}} \mright) = \mathcal D \mleft( \mathcal F^{-1} |\xi| \mathcal F \mright) = \mleft\{ \psi \in L^2 \mleft(\mathbb{R}^3 \mright) \, \middle| \, \int_{\mathbb{R}^n} \mleft(1+|\xi|\mright)^2 \mleft| \hat\psi (\xi) \mright|^2 < \infty \mright\} =	\\&= H^1 \mleft( \mathbb{R}^n \mright)
	\end{align*}
\end{example*}

\begin{defn}

	For $A \geqslant 0$ let $\Braket{|A|^{\nicefrac{1}{2}}\psi , |A|^{\nicefrac{1}{2}} \psi} = \Braket{\psi,A\psi}$ is the quadratic form associated to $A$.
	
	Thus $\mathcal Q(A)$ is the natural domain of the quadratic form.

\end{defn}

\begin{remark}[Question]

If we have a quadratic form, does this maybe determine a self-adjoint
operator? For matrices this is the case: For vectors $v$ and $w$, you
can compute every matrix element $\langle v, Aw\rangle$ from the
knowledge of $\langle v\pm w,A(v\pm w)\rangle$ and $\langle v\pm iw,
A(v\pm iw)\rangle$ as you can check by expanding.

\end{remark}

\begin{theorem}[Friederichs]
	Suppose that $A: \mathcal D(A) \subset \mathcal G \rightarrow  \mathcal G$ is symmetric.
	
	If there exists a constant $c \in \mathbb{R}$ such that $A \geqslant c$, then there exists a distinguished (only one) self-adjoined extension $A_{\text{ext}}$ of $A$ such that $A_{\text{ext}} \geqslant c$ and
	\[
		\mathcal Q(A_{\text{ext}})=\mathcal  Q \mleft( A_{\text{ext}} - c +1 \mright) \supset \mathcal Q(A).	
	\]
	where we mean by extension that for all $\psi \in \mathcal D(A): A_{\text{ext}} \psi = A \psi$. 
\end{theorem}

\begin{example*}
\begin{enumerate}

	\item[1)] Let $\mu^{\text{th}}$ partial derivative be denoted by 
	\[
		\partial_{\mu}: \mathscr C^{\infty}_0 \mleft( \mathbb{R}^n \mright) \longrightarrow L^2 \mleft( \mathbb{R}^n \mright)	
	\]
	and further consider the operator 
	\[
		A:= - \partial_\mu^2 = \mathscr C_0^{\infty} \mleft( \mathbb{R}^n \mright) \longrightarrow L^2 \mleft( \mathbb{R}^n \mright)
	\]
	To check that $A$ is positive let $\phi,\psi \in \mathscr C_0^{\infty} \mleft( \mathbb{R}^n \mright)$ and calculate
	\[
		\Braket{\phi, A \psi} = \Braket{\phi, - \partial_\mu^2 \psi} = \int\limits_{\mathbb{R}^n} \overline{\phi(x)} (-1) \mleft( \partial_\mu^2 \psi \mright) (x) \d x = \int\limits_{\mathbb{R}^n} \overline{\partial_\mu \phi(x)} \partial_\mu \psi(x)	 \d x,
	\]
	hence $\Braket{\psi, A \psi} \geqslant 0$, therefore $A \geqslant 0$ and thus symmetric.
	
	Friedrichs theorem now tells us that there exists a unique self adjoint extension of $A_{\text{ext}} \geqslant 0$ of $\partial_\mu^2$. It is an exercise to show that 
	\[
		A_{\text{ext}} =\mathcal F^{-1}\mleft( |\xi|^2\hat\psi \mright)
	\] 
	
        Thus we obtain the appropriate self-adjoint extension of the Laplacian by
        ``defining it via the Fourier transform'' on $H^2$.
	\item[2)] Let $H = - \Delta - \lambda \chi_B$ for $\lambda >0$
          and $\chi_B$ the characteristic function of some
          (measurable) set $B\subset \mathbb{R}^3$. This operator is obviously defined on $\mathscr C^{\infty}_0\mleft( \mathbb{R}^n \mright)$, since \\ $\int |-\Delta \psi|^2 \d x < \infty$ and $\lambda^2 \int_B |\psi|^2 < \infty$. Further it is symmetric and 
	\[
		H \geqslant - \Delta - \lambda,	
	\]
	because 
	\[
		\Braket{\psi, H \psi} = \int \mleft|\nabla \psi \mright|^2 \d x - \lambda \int\limits_B |\psi|^2 \d x	\geqslant \Braket{\psi, (- \Delta-\lambda) \psi},
	\]
	hence we may use Friedrichs Theorem.
	\item $H = - \Delta + c |x|^2\geqslant 0$, for $c \geqslant 0$. This operator is positive and therefore extendable by Friedrich
	\item $H_h = -\Delta - \frac{z}{|x|}$. 
	
	We claim that $H_h$ is symmetric on $\mathscr C^{\infty}_0 \mleft( \mathbb{R}^3 \mright)$. To prove this let $\psi \in \mathscr C_0^{\infty}$.
	\[
		\int\limits_{\mathbb{R}^3} |-\Delta \psi|^2 \d x = \int\limits_{|x| \leqslant M} |-\Delta \psi|^2 \d x \leqslant \|-\Delta \psi \|^2_{\infty} \int\limits_{|x|\leqslant M} \int \d x = 	\|-\Delta \psi \|_{\infty}  \frac{4\pi}{3} M^3,
	\]
	for $M > 0$ such that $\supp \psi \subset B_R$. Further we have 
	\[
		\int\limits_{\mathbb{R}^3}\mleft| - \frac{z}{|x|} \psi \mright|^2 = \int\limits_{|x|\leqslant M} \mleft| - \frac{z}{|x|} \psi(x) \mright|^2 \le z^2 \|\psi\|_{\infty}^2 \int\limits_{|x| \leqslant M} \frac{1}{|x|^2} \d x = 4 \pi  z^2 \|\psi\|_{\infty}^2 < \infty.
	\]
	Obviously $\psi$ is symmetric since 
	\[
		\Braket{\psi, H_h \psi} = \int \mleft( \bar \psi (-\Delta \psi) - \frac{z}{|x|}|\psi|^2 \mright) \d x	
	\]
	To show that $\psi$ is bounded from below we will need the following lemma.	
	
	Applying that to $H_h$ we get 
	\begin{align*}
		\mathcal E[\psi] &= \int\limits_{\mathbb{R}^3} \mleft( |\nabla \psi|^2 - \frac{z}{|x|}|\psi|^2 \mright) \d x 	= \int\limits_{|x|< R } \mleft( |\nabla \psi|^2 - \frac{z |x|}{|x|^2}|\psi|^2 \mright) \d x + \int\limits_{|x|\geqslant R} \mleft( |\nabla \psi|^2 - \frac{z}{|x|}|\psi|^2 \mright) \d x \geqslant \\
		& \geqslant \int\limits_{|x|< R} \mleft( |\nabla \psi|^2 - \frac{z R}{|x|^2}|\psi|^2 \mright) \d x + \int\limits_{|x|\geqslant R} |\nabla \psi|^2 \d x - \frac{z}{R} \int\limits_{\mathbb{R}^3} |\psi|^2 \d x \stackrel{R = \frac{1}{4z}}{=\joinrel=}\\
		&= \int\limits_{\mathbb{R}^3} |\nabla \psi|^2 -  \frac{1}{4} \int\limits_{|x|< \frac{1}{4z}} \frac{1}{|x|^2} |\psi(x)|^2 \d x - 4 z^2 \int\limits_{\mathbb{R}^3} |\psi|^2 \d x \geqslant \\
		& \geqslant \int\limits_{\mathbb{R}^3} \mleft( |\nabla \psi|^2 - \frac{1}{4}\frac{1}{|x|^2}|\psi(x)|^2 \mright) \d x - 4 z^2 \int\limits_{\mathbb{R}^3} |\psi|^2 \d x \geqslant -16 \frac{z^2}{4} \|\psi\|^2,
	\end{align*}
	hence by Friedrichs thoerem we can extend $H_h$.
	
\end{enumerate}
\end{example*}

\begin{lemma}[Hardy Inequality or Quantum Mechanical uncertainty principle]
	\[
		\int\limits_{\mathbb{R}^3} |\nabla\psi|^2 \d x \geqslant \frac{1}{4} \int\limits_{\mathbb{R}^3} \frac{|\psi(x)|^2}{|x|^2} \d x 	
	\]
	for $\psi \in \mathscr S\mleft( \mathbb{R}^3 \mright)$.
\end{lemma}

\begin{example*}[Relativistic Hamiltonians]
	Using the relativistic kinetic energy expression $E = \sqrt{c^2 p^2 + m^2 c^4}$ we can define the relativistic Hamiltonian to be 
	\[
		\sqrt{p^2c^2 + m^2c^4} \psi := \mathcal F^{-1} \mleft( \sqrt{c^2 \hbar^2 \xi^2 + m^2c^4} \hat\psi \mright)	
	\]
	for $\psi \in \mathscr S \mleft( \mathbb{R}^n \mright)$.
	
	We now define the Chandrasekhar Hydrogen Hamiltonian to be 
	\[
		C_h = \sqrt{c^2 p^2 + m^2 c^4} - \frac{z}{|x|} \cong m c^2 \mleft( \sqrt{p^2 +1} - \frac{z \alpha}{|x|} \mright),	
	\]
	with $\alpha:= \frac{1}{\hbar c}$. From now on we shall use units such that $c = \hbar = 1$, i.e.
	\[
		C= \sqrt{p^2 +1} - \frac{\gamma}{|x|}.
	\]
	using the following lemma we get 
	\[
		|p| +1 \geqslant C - \sqrt{p^2 +1} \geqslant |p|,	
	\]
	Boundedness from below of $\mathscr C$ therefore is equivalent to showing that 
	\[
		|p| - \frac{\gamma}{|x|}	
	\]
	is bounded from below.
\end{example*}

\begin{lemma}[Kato, Herbst]
For $\psi \in \mathscr S \mleft( \mathbb{R}^3 \mright)$ we have the inequality 
\[
	\int\limits_{\mathbb{R}^3} |\xi| \mleft| \hat\psi(\xi) \mright|^2 \d x \geqslant \frac{2}{\pi} \int\limits_{\mathbb{R}^3} \frac{1}{|x|}|\psi(x)|^2 \d x
\]
\end{lemma}
This provides boundedness from below only for small enough $Z$.

\begin{example*}
	For the Chandrasekar operator 
	\[
		C = \mleft( c^2 p^2 + \mleft( m c^2 \mright)^2 \mright)^{\nicefrac{1}{2}} - \frac{Z e^2}{|x|}	
	\]
	we have shown that for all $\psi \in \mathscr S \mleft( \mathbb{R}^3 \mright)$
	\begin{align*}
		\mathcal E [\psi] = \int\limits_{\mathbb{R}^3} \mleft( \xi^2 c^2 \hbar^2 + m^2c^4 \mright)^{\nicefrac{1}{2}} |\psi(s)|^2 - \frac{Ze^2}{2\pi^2} \int\limits_{\mathbb{R}^3} \frac{1}{|\xi-\xi'|^2} \overline{\psi(\xi)} \psi(\xi') \d\xi\d\xi' \geqslant 0
	\end{align*}
	iff $Z \frac{e^2}{\hbar c} = Z \alpha := \gamma \leqslant \frac{2}{\pi}$. This is equivalent to for all $\psi \in \mathscr S \mleft( \mathbb{R}^3 \mright)$
	\begin{align*}
		\int |\xi||\psi(\xi)|^2 \d \xi \geqslant \frac{\gamma}{2\pi^2} \int \frac{1}{|\xi-\xi'|^2} \overline{\psi(\xi)}\psi(\xi') \d \xi \d \xi' \quad \text{iff } \gamma \leqslant \frac{2}{\pi}
	\end{align*}.
	Let
	\[
		(V\psi)(\xi') = \frac{1}{2\pi^2} \int\limits_{\mathbb{R}^3} \frac{1}{|\xi-\xi'|} \psi(\xi') \d \xi',	
	\]
	with
	\[
		\frac{2}{\pi} \mleft( V \frac{1}{|\cdot|^2}\mright)(\xi') = \frac{|\xi|}{|\xi|^2}.	
	\]
	The quadratic form corresponding to $C$ is 
	\begin{align*}
		Q[\psi] &= \int |\xi| |\psi(s)|^2 - \frac{2}{\pi} \frac{1}{2\pi^2} \int \frac{1}{|\xi-\xi'|^2} \overline{\psi(\xi)} \psi(\xi') \d\xi\d\xi'.
	\end{align*}
	Let $\psi = f g $, where $g(\xi) = \frac{1}{|\xi|^2}$, then we can rewrite the quadratic form as 
	\begin{align*}
		Q[\psi] &= \int |\xi| |\psi(\xi)|^2\d \xi - \frac{1}{2\pi^3} \int \frac{1}{|\xi-\xi'|^2} g(\xi)g(\xi') 2\Re (f(\xi) f(\xi')) \d\xi\d\xi' = \\
		&=\int |\xi| |\psi(\xi)|^2\d \xi - \frac{1}{2\pi^3} \int \frac{g(\xi)g(\xi')}{|\xi-\xi'|^2}  \mleft(|f(\xi)-f(\xi')|^2 - |f(\xi)|^2-|f(\xi')|^2 \mright) \d\xi\d\xi' =\\
		&=\int |\xi| |g(\xi)|^2|f(\xi)|^2\d \xi - \int g(\xi') |f(\xi')|^2 \frac{1}{2} (Vg)(\xi) \d\xi' - \int  g(\xi)|f(\xi)| \frac{1}{2} (Vg)(\xi) \d \xi + \\
		&\quad + \frac{1}{2\pi^3} \int \frac{g(\xi)g(\xi')}{|\xi-\xi'|^2} |f(\xi)-f(\xi')|^2 \d \xi \d \xi' \stackrel{\footnotemark}{=}\\
		&= \frac{1}{2\pi^3} \int \frac{g(\xi)g(\xi')}{|\xi-\xi'|} |f(\xi)-f(\xi')|^2\d \xi \d \xi'  \geqslant 0
	\end{align*}
	\footnotetext{Where we used the expression for $\mleft( V \frac{1}{|\cdot|} \mright)$ with which the extra terms cancel (renaming $\xi \leftrightarrow \xi'$)}
	Moreover $Q[\psi] = 0$ occurs only for $f$ constant, however $fg$ for $f \neq 0$ is not in $\mathscr S$, so $f =0$. Thus $\mathcal E[\psi] \geqslant 0$ if $\gamma \leqslant\frac{2}{\pi}$ which implies that $C$ can be extended due to Friedrichs to a self-adjoint operator.
	
	Note that the argument fails for $\gamma > \frac{2}{\pi}$. The homework will be to find that $\psi \in \mathscr S$ such that $Q[\psi] < 0$ and conclude that it is false that there exists $M$ such that for all $\psi \in \mathscr S$, $\mathcal E[\psi]\geqslant M \|\psi\|^2$.
	
	(Physically this means that the particel is eaten by the singularity.)
\end{example*}


\section{Perturbations}

\begin{theorem}[Kato \& Rellich]
	Let $\mathcal H$ be a Hilbert space, and $A: \mathcal D(A) \subset \mathcal H \rightarrow \mathcal H$ a self-adjoint operator. Let $B : \mathcal D(B) \rightarrow \mathcal{H}$ be a symmetric operator with $\mathcal D(B) \supset \mathcal D(A)$, and let there exist $a<1$ and $b \in \mathbb{R}$ such that for all $\psi \in \mathcal D(A)$ holds
	\[
		\|B\psi\| \leqslant a \|A\psi\| + b \|\psi\|.	
	\]
	Then 
	\[
		A+B:	
		\begin{aligned}
			\mathcal D(A) & \longrightarrow \mathcal H\\
			\psi & \longmapsto A\psi + B\psi
		\end{aligned}	
	\]
	is self-adjoint.
\end{theorem}
\begin{remark*}
	The hypothesis of the above theorem is called $B$ is $A$-bounded with relative bound $a$.
\end{remark*}

\begin{example*}
	Consider the hydrogen Hamiltonian 
	\[
		H = - \Delta - \frac{Z}{|x|}.	
	\]
	We have already proven that $H$ extends to a self-adjoint operator according to Friedrichs, with quadratic from domain
	\[
		\mathcal Q(H) \supset \mathscr S\mleft( \mathbb{R}^3 \mright)
	\]
	Let $B$ be the potential operator, then the question is whether we can show that 
	\begin{align*}
		\|B\psi\| &= \mleft( \int\limits_{\mathbb{R}^3} \mleft| \frac{Z}{|x|} \psi(x) \mright|^2 \mright)^{\nicefrac{1}{2}} = Z 	\mleft( \int \frac{|\psi(x)|^2}{|x|^2} \d x \mright)^{\nicefrac{1}{2}} \leqslant \\ & \stackrel{?}{\leqslant} a \mleft( \int\limits_{\mathbb{R}^3} |(-\Delta \psi)(x)|^2 \d x \mright)^{\nicefrac{1}{2}} + b \mleft( \int\limits_{\mathbb{R}^3} |\psi(x)|^2 \mright)^{\nicefrac{1}{2}}.
	\end{align*}
	Let us decompose the potential into 
	\[
		\frac{Z}{|x|} = V_{<}(x)+ V_{>}(x) = \frac{Z}{|x|} \chi_{B_R}(x)	+\frac{Z}{|x|} \chi_{B_R^C}(x),	
	\]
	where $V_<$ has compact support but also singularity and $V_>$ does not have compact support and no singularity. For this decomposition we get
	\[
		\|B\psi\| = \mleft\| V_< \psi + V_> \psi \mright\| \leqslant \mleft\| V_< \psi \mright\| + \mleft\| V_{>} \psi \mright\| \leqslant \mleft\|V_{<} \psi \mright\| + \frac{Z}{R} \|\psi\|,	
	\]
	So, $\frac{Z}{|x|}$ is $-\Delta$ -bounded with constant $a$, if and only if $V_<$ is. Thus we have reduced our inquiry to
	\[
		\mleft\| V_< \psi \mright\| \stackrel{?}{\leqslant} a \|\Delta \psi\| + b \|\psi\|.	
	\]  
	We have \[
		\|\Delta \psi\| = \| (-\Delta + 1) - 1 \psi \| \geqslant \|(-\Delta +1) \psi\| - \|\psi\|,
	\]
	thus it is enough to show that there exists a $a< 1 $ such that for all $\psi \in \mathcal D(-\Delta)$ 
	\[
		\|V_< \psi \| \leqslant a \|(-\Delta + 1) \psi\| + b \|\psi\|	
	\]
	Let us write $\psi = (-\Delta + 1)^{-1} \phi$ (as $\phi$ runs through $\mathcal H$, $\psi$ runs through $H^2 \mleft( \mathbb{R}^3 \mright)$).
	\[
	 	\mleft\|V_{<} \mleft( -\Delta + 1 \mright)^{-1} \phi \mright\| \leqslant a \|\phi\|^2 + b \mleft\| (-\Delta +1)^{-1}\phi \mright\|.	
	\]
	
	The integral kernel of $(-\Delta +1)^{-1}$ is given by	
	\[
		(-\Delta +1)^{-1}(x,y) = \frac{e^{-|xy|}}{4\pi|x-y|}	,
	\]
	i.e. 
	\[
		\mleft(\mleft( -\Delta+1 \mright)^{-1} \psi \mright)(y) = \int \frac{e^{-|x-y|}}{4\pi|x-y|} \psi(x) \d x .
	\]
	We have to calculate
	\begin{align*}
		\int \mleft| V_{<}(y) \int \frac{e^{-|x-y|}}{4\pi|x-y|} \phi(x) \d x \mright|^2 \d y &\leqslant \int \mleft|V_{<}(y) \mright|^2 \mleft| \int \frac{e^{-|x-y|}}{4\pi|x-y|} \phi(x) \d x \mright|^2  \d y \leqslant \\
 & \leqslant 		\int \mleft|V_{<}(y) \mright|^2 \mleft| \mleft( \int \frac{e^{-2|x-y|}}{(4\pi)^2|x-y|^2} \d x \mright)^{\nicefrac{1}{2}}  \mleft( \int |\phi(x)|^2 \d x   \mright)^{\nicefrac{1}{2}} \mright|^2 \d y = \\
 	&=\int \mleft|V_{<}(y) \mright|^2 \mleft( \mleft( \int \frac{e^{-2|x|}}{(4\pi)^2|x|^2} \d x \mright)\mleft( \int |\phi(x)|^2 \d x   \mright) \mright)\d y = \\
 	&=\int \mleft|V_{<}(y) \mright|^2 \d y  \underbrace{\mleft( \int \frac{e^{-2|x|}}{(4\pi)^2|x|^2} \d x \mright)}_{=:c}\mleft( \int |\phi(x)|^2 \d x   \mright) = \\
 	&= c \|\phi\|^2 \int\limits_{|x| < R} \frac{Z^2}{|x|^2} \d x = c \|\phi\|^2 4\pi R Z^2,
	\end{align*}
	Since $R$ we can make $R$ arbitrarily small we have that
	\[
		\mleft\|V_{<} \mleft(-\Delta + 1\mright) \phi\mright\| \leqslant \epsilon\|\phi\| + 0\|\phi\|,	
	\]
	for $\epsilon^2 = 4\pi c RZ^2$.
	
	Thus we have for all $\psi \in \mathcal H$ 
	\[
		\mleft\| \frac{1}{|\cdot|}\psi\mright\| \leqslant a \|-\Delta \psi \| + b \|\psi\|,	
	\]
	holds for arbitrarily small but positive $a$ (while $b$ depends on $a$). One says that $B = \frac{Z}{|\cdot|}$ is infinitestessimally bounded.
	
	From Kato, Relich we get 
	\[
		\mathcal D(-\Delta - \frac{Z}{|x|}) = H^2 \mleft( \mathbb{R}^3 \mright)	
	\]
	
	

	
\end{example*}


\begin{theorem}[KLMN]
	Suppose that $A, B$ are symmetric and that $\mathcal Q \mleft( B \mright) \supset \mathcal Q (A)$ and that there exist $a<1$ and $b \in \mathbb{R}$, such that 
	\[
			q_B[\psi] = \Braket{\psi, B \psi} \leqslant a \Braket{\psi, A \psi} + b \Braket{\psi,\psi},
	\]
	The form $\Braket{\psi, A \psi} + \Braket{\psi, B\psi}$ defines a self-adjoint operator (which is equivalent to the Friedrichs extension $A \dot+ B$ of $A+B$ and $\mathcal Q \mleft( A \dot + B \mright) = \mathcal Q(A)$).
	
\end{theorem}

\begin{remark}
	Thus information on the form domain is gained.
\end{remark}

\begin{example*}
	Consider the operator
	\[
		D_{\text{FW}} : \begin{aligned}
			H^1 \mleft( \mathbb{R}^3, \mathbb{C}^4 \mright) & \longrightarrow L^2 \mleft( \mathbb{R}^3, \mathbb{C}^4 \mright)\\
			\psi & \longmapsto \begin{pmatrix}
			e(\xi) &&&\\
			&e(\xi) &&\\
			&&-e(\xi) &\\
			&&&-e(\xi)
\end{pmatrix} \psi = \begin{pmatrix}
				e(\xi) \psi_1(\xi)\\
				e(\xi) \psi_2(\xi)\\
				-e(\xi)\psi_3(\xi)\\
				-e(\xi)\psi_4(\xi)
			\end{pmatrix}
		\end{aligned}	
	\]
	where 
	\[
		e(\xi) = \sqrt{c^2 \xi^2 + m^2 c^4}	.
	\]
	Further let $F_W(\xi)$ be the matrix such that 
	\[
		\hat D(\xi) = c \bm \alpha \cdot \xi + m c^2 \beta = F_W(\xi) D_{\text{FW}}(\xi) F_W(\xi)^*,	
	\]
	where 
	\[
		\bm \alpha = \begin{pmatrix}
		0 & \bm \sigma\\
		\bm \sigma & 0
\end{pmatrix}, \quad \beta = \begin{pmatrix}
1 & 0\\ 0 &-1
\end{pmatrix}
	\]
	with $\bm \sigma$ denoting the three Pauli matrices
	\[
		\sigma_1 = \begin{pmatrix}
			0&1\\1&0
\end{pmatrix}	, \qquad \sigma_2 = \begin{pmatrix}
			0&-i\\i&0
\end{pmatrix}, \qquad \sigma_3 = \begin{pmatrix}
			1&0\\0&-1
\end{pmatrix}					
	\]
	The transformation $D(\xi) \rightarrow \hat D(\xi)$ is called the Foldy-Wouthuysen Transform.
	
	To find $F_W(\xi)$ first the matrices $\alpha_i$ are unitary, since $\sigma_i$ are. Let $\bm \gamma = (\alpha_1, \alpha_2, \alpha_3, \beta)$.

	We can write $\hat D(\xi) = e(\xi) \cdot \sum_{j =1}^4 \omega_j \gamma_j$, where $|\omega| =1$. Note that $\sum_{j =1}^4 \omega_J \gamma_j$ has only eigenvalues $\pm 1$ each with multiplicity 2, since $\Tr \hat D(\xi) = \Tr D_{\text{FW}}(\xi) = 0$.
	
	Thus $\hat D(\xi)$ is a self-adjoint operator (real-valued multiplication operator) from $H^1 \mleft( \mathbb{R}^3, \mathbb{C}^4 \mright)$. The operator
	\[
		D_0 = \mathcal F \hat D \mathcal F^{-1},	
	\]
	is called the \bitext{free Dirac operator}, which is an extension of 
	\[
		\sum_{j = 1}^3 c \alpha_j \frac{1}{i} \partial_j + mc^2 \beta.
	\]
	
	The spectrum of the free Dirac operator is $\sigma \mleft( D_0 \mright) = (-\infty, -mc^2]\cup [ mc^2,\infty)$.
	
	
	The Dirac operator for the Hydrogen atom is given by 
	\[
		D_Z = D_0 - \frac{Z e^2}{|x|}.	
	\]
	We want to check that 
	\[
		\mleft\| \frac{Ze^2}{|x|} \psi \mright\| \leqslant a \|D_0 \psi \| + b\|\psi\|.	
	\]
	\begin{align*}
		Z e^2 \sqrt{\int_{\mathbb{R}^3} \frac{1}{|x|^2} |\psi(x)|^2 } &\stackrel{?}{\leqslant} a \overbrace{\sqrt{\Braket{D_0 \psi, D_0 \psi}}}^{ = \sqrt{ \int \mleft| \hat D(\xi) \hat \psi (\xi) \mright|^2 \d \xi}} +  b \|\psi\| = a \sqrt{\int |e(\xi)|^2 \mleft| \hat\psi(\xi) \mright|^2 \d \xi} + b \|\psi\| = \\
		&= a \sqrt{ \int \mleft( c^2 |\nabla \psi|^2 + m^2 c^4 |\psi|^2 \mright) \d x} + b \sqrt{\int|\psi|^2 \d x}.
	\end{align*}
	Hence this requires 
	\[
		\mleft(Ze^2 \mright)^2 \int \frac{|\psi(x)|^2}{|x|^2} \d x \leqslant a^2 c^2 \int |\nabla \psi|^2 \d x	
	\]	
	Using Hardy's inequality we get that 
	\[
		Z^2 \frac{e^4}{c^2} < \frac{1}{4} \implies Z \alpha_s < \frac{1}{2},	
	\]	
	where $\alpha_S = \frac{e^2}{\hbar c}$ is the Sommerfeld fine structure constant. This condition is also necessary for the inequality to be satisfied.
	
	This works for $Z < \frac{137.037}{2} \approx 69$ (Further extensions: see Nenciu for $Z\alpha < 1$).
	
\end{example*}


\begin{example*}
	Let $A = -\Delta$, with $\mathcal Q \mleft( A \mright) = H^1 \mleft( \mathbb{R}^3 \mright)$. Now consider the quadratic form 
	\[
		\Braket{\psi, B \psi} = \int \frac{\gamma}{|x|^2} |\psi(x)|^2 \d x,	
	\]
	for $\gamma \geqslant 0$.
	
	We get that 
	\[
		\gamma 	 \int \frac{1}{|x|^2} |\psi(x)|^2 \d x \leqslant a \int|\nabla \psi|^2 \d x + b \int |\psi|^2 \d x \iff \frac{\gamma}{\alpha}\leqslant \frac{1}{4},
	\]
	since we just need $a < 1$. In this case $-\Delta - \frac{\gamma}{|\cdot|}$ is defined by its quadratic form and its form domain is $H^1 \mleft( \mathbb{R}^3 \mright)$.
\end{example*}

\begin{remark*}
	\begin{enumerate}
		\item[1)] For $\gamma \geqslant \frac{1}{4}$ the inequality is not satisfied.
		\item[2)] $\gamma = \frac{1}{4}$ ensures that $\int |\nabla \psi|^2 \d x - \gamma \int|\frac{\psi|^2}{|x|^2} \d x$ is still positive, so Friedrich's Theorem still gives us a self-adjoint operator.
	\end{enumerate}
\end{remark*}


\section{Multi-Particle Operators}.

Consider the Hamiltonian
\[
	H = \sum_{n = 1}^N \mleft( - \Delta_n - \phi \mleft( x_n \mright) \mright) + \sum_{1 \leqslant n < m \leqslant N} \frac{1}{\mleft| x_n - x_m \mright|} + \sum_{1 \leqslant k < j \leqslant K} \frac{Z_j Z_k}{\mleft| R_k - R_j \mright|},
\]
where 
\[
	\phi(x) = \sum_{k = 1}^K \frac{Z_k}{|x-R_k|}
\]


The wave function describing this system must be of the form $\psi \mleft( x_1, \dots, x_N \mright)$ antisymmetric and a spinor. Since the one particle Hilbert space is given by $\mathcal H = L^2 \mleft( \mathbb{R}^3 , \mathbb{C}^{2s + 1} \mright) $, where $s$ is the spin of the particle, the multiparticle state is in 
\[
	\mathcal H^{(N)}:= \bigwedge_{n=1}^N L^2 \mleft( \mathbb{R}^3, \mathbb{C}^{2s_n + 1} \mright),
\]
which in particular has $\prod_{i = 1}^N (2 s_n +1)$ components and is a function of $3N$ variables.

Let $\psi \in \mathscr S \mleft( \mathbb{R}^{3N}, \mathbb{C}^{(2s +1)^N}\mright) $ (i.e. every particle has the same spin), i.e. the antisymmetric Schwartz space, which is dense in the Hilbert space. We get
\begin{align*}
	\Braket{\psi, H \psi} &= \sum_{n=1}^N \mleft( \int \mleft| \nabla_n \psi \mright|^2 \d x -  \int |\psi|^2 \sum_{k = 1}^N \frac{Z_k}{|x-R_k|} \d x + \int \sum_{1 \leqslant n < m \leqslant N} \frac{1}{|x_n-x_m|} |\psi|^2 \d x + R \|\psi\|^2 \mright)= \\
	&\geqslant \sum_{n=1}^N \sum_{k =1}^K \underbrace{ \int \mleft( \frac{1}{K} |\nabla_n \psi|^2- \frac{Z_k}{|x_n - R_k|} |\psi|^2 \mright)}_{ \geqslant - d Z^2_k K \|\psi\|^2} \d x \geqslant - N K \sum_{j = 1}^K Z_j^2 \cdot \text{const},
\end{align*}
where we used the gound state energy of the Coulomb potential, i.e.
\[
	- \Delta - \frac{Z}{|\cdot|} \geqslant -\frac{1}{4} Z^2
\]

Thus Friedrich's Theorem gives a self-adjoint extension of the above.


Recall \[
	H_{N,\mathcal R,\mathcal Z} = \sum_{n = 1}^N \mleft( - \Delta_n - \phi \mleft( x_n \mright) \mright) + \sum_{1 \leqslant n < m \leqslant N} \frac{1}{\mleft| x_n - x_m \mright|} + \sum_{1 \leqslant k < j \leqslant K} \frac{Z_j Z_k}{\mleft| R_k - R_j \mright|},
\]
where $R_k$ are point-wise different such as in $\mathcal H^{(N)} = \bigwedge_{n=1}^N L^2 \mleft( \mathbb{R}^3, \mathbb{C}^q \mright)$.

\subsection*{Thomas-Fermi-Theory}

Consider 
\[
	\mathcal C:= \mleft\{ \rho \, \middle| \, \rho \text{ measurable, } \frac{1}{2} \int \int \frac{|\rho(x)\rho(y)|}{|x-y|} \d x \d y < \infty \mright\}.
\]
in particular we define the function on $\mathcal C$
\[
	D[\rho] =  \frac{1}{2} \int \int \frac{|\rho(x)\rho(y)|}{|x-y|} \d x \d y.
\]


$\mathcal C$ is a Hilbert space with scalar product
\[
	\Braket{\rho,\sigma} = \frac{1}{2} \int \int \frac{\overline{\rho(x)}\sigma(y)}{|x-y|} \d x \d y  
\]
since 
\[
	\int \int \frac{\overline{\rho(x)}\rho(y)}{|x-y|} \d x \d y = c \int \frac{\mleft| \hat\rho(\xi)\mright|^2}{|\xi|^2} \d \xi > 0, \quad \text{unless } \rho = 0 \;\text{a.e.},
\]
hence this defines a positive-definite sesquilinear form, i.e. a scalar product.

Let
\[
	\mathcal D:= \mleft\{ \rho \geqslant 0 \, \big| \, \rho \in L^{\nicefrac{5}{3}} \mleft( \mathbb{R}^3 \mright) \cap \mathcal C \mright\}
\]
on which we define the Thomas-Fermi-Functional
\[
	\mathcal E_{\text{TF}}: \begin{aligned}
		\mathcal D & \longrightarrow \mathbb{R}\\
		\rho & \longmapsto \int\limits_{\mathbb{R}^3} \mleft( \frac{3}{5} \gamma_{\text{TF}}  \rho(x)^{\nicefrac{5}{3}} - \phi(x)\rho(x) \mright) \d x + D[\rho],
	\end{aligned}
\]
where $\gamma_{\text{TF}}:= \mleft( \frac{6 \pi^2}{q} \mright)^{\nicefrac{3}{2}}$.

It remains to show that $\int \phi(x)\rho(x) \in \mathbb{R}$. For this, since 
$
	\phi(x) = \sum_{k = 1}^N \frac{Z_k}{\mleft| x- R_k \mright|}
$, it is enough to show that 
\[
	\int \frac{\rho(x)}{|x-R|} \d x < \infty
\]
for all $R$ and all $\rho \in \mathcal D$. For this it is, by change of variable, enough to show that 
\[
	\int\frac{\rho(x)}{|x|} \d x < \infty
\]
for all $\rho \in \mathcal D$.

Since $\frac{1}{|x|} = W_1(x)+W_2(x)$, with
\[
	W_1(x) = \int\limits_{|y|\leqslant 1} \frac{1}{|x-y|} \frac{\d y}{\frac{4 \pi}{3}} = \int \frac{\sigma(y)}{|x-y|} \d y, \quad \sigma(y) = \frac{3}{4 \pi} \chi_{B_1(0)}(y)
\]
using Newton's Shell Theorem we get 
\[
	W_1(x) = \begin{cases}
		?, \qquad & \text{if } |x|\leqslant 1\\
		\frac{1}{|x|}, & \text{if } |x| \geqslant 1
	\end{cases}
\]
in particular it follows that $W_2(x)$ has compact support in $B_1(0)$.

	Again using Newton's shell theorem we get for $|x|\leqslant 1$ 
	\[
		W_1(x) = \int\limits_{0}^1 \frac{4\pi r'}{\max\{r',|x|\}} \frac{3}{4\pi}	\d r' \leqslant \frac{1}{|x|} \int\limits_0^1 3 r'^2 = \frac{1}{|x|}
	\]
	therefore we also get $W_2 \geqslant 0$.

	Thus we get \[
		\int\limits_{\mathbb{R}^3} W_2(x) \rho(x) \d x \leqslant \int\limits_{|x|\leqslant 1} \frac{\rho(x)}{|x|} \d x \stackrel{\text{H\"older}}{\leqslant} \mleft( \int\limits_{|x| \leqslant 1} \rho(x)^{\nicefrac{5}{3}} \mright)^{\nicefrac{3}{5}} \mleft( \int\limits_{|x| \leqslant 1} \frac{1}{|x|^{\nicefrac{5}{2}}}  \d x\mright)^{\nicefrac{2}{5}} < \infty	
	\]
	Further we also have 
	\[
		\frac{1}{2} \int\limits_{\mathbb{R}^3} \rho(x) W_1(x) \d x = \frac{1}{2} \int \rho(x) \int \frac{\sigma(y)}{|x-y|} \d y \d x \stackrel{\text{Schwarz}}{\leqslant} \underbrace{\sqrt{D[\rho]}}_{< \infty} \sqrt{D[\sigma]}
	\]
	Since 
	\begin{align*}
		\text{const} D[\sigma] &= \int\limits_{|x|<1} \int\limits_{|y|<1}	\frac{1}{|x-y|} \d y \d x= \int\limits_0^1 \int\limits_0^1 \frac{4 \pi r^2 4 \pi r'^2}{\max\{r,r'\}} = \text{const} \int\limits_0^1 \d r'^2  \int\limits_0^1 \frac{r^2}{\max\{r,r'\}}  \d r \d r' = \\
		&= \text{const} \int\limits_0^1 r'^2 \mleft( \int\limits_0^{r'} \frac{r^2}{r'} + \int\limits_{r'}^1 \frac{r^2}{r} \d r \mright) \d r' = \text{const} \mleft( \int\limits_0^1 r' \frac{1}{3} r' \d r' + \int\limits_0^1 \frac{1}{2} (1-r'^2) \d r' \mright) < \infty
	\end{align*}
	hence $\mathcal E_{\text{TF}}$ is well-defined


\begin{lemma}
	The infimum of $\mathcal E_{\text{TF}} \mleft( \mathcal D \mright) >-  \infty$
\end{lemma}

\begin{proof}
	\begin{align*}
		\int \phi(x) \rho(x) \d x &= \sum_{k = 1}^N \int \mleft( Z_k W_1 \mleft( x- R_k \mright) + Z_k W_2 \mleft( x- R_k \mright) \mright) \rho(x) \d x \leqslant\\
		&\leqslant \underbrace{\sum_{k = 1}^N Z_k}_{=: Z_{\text{tot}}} \mleft( \|\rho\|_{\nicefrac{5}{3}} \|W_2\|_{\nicefrac{5}{2}}+ \sqrt{D[\rho]} \sqrt{D[\sigma]} \mright),
	\end{align*}
	Hence we get 
	\[
		\mathcal E_{\text{TF}}[\rho] \geqslant \frac{3}{5} \gamma_{\text{TF}} \|\rho\|_{\nicefrac{5}{3}}^{\nicefrac{5}{3}} -  Z_{\text{tot}} \|\rho\|_{\nicefrac{5}{3}}	\|W_2\|_{\nicefrac{5}{2}} - Z_{\text{tot}} \sqrt{D[\rho]} \sqrt{D[\sigma]} + D[\rho] \geqslant-M > -\infty	
	\]
\end{proof}

\begin{lemma}
\label{MinEx}
	There exists a unique $\rho_{\min}$ in $\mathcal D$ such that $\mathcal E_{\text{TF}} \mleft[ \rho_{\min} \mright] = \inf \mathcal E(\mathcal D)$.
\end{lemma}

\begin{lemma}[Banach-Alaoglu]
	If $X$ is a reflexive Banach Space (i.e. $X'' = X$) and $\mleft( x_n \mright)_n \subset X$ is bounded then 
	\[
		\exists \mleft( x_{n_{\nu}} \mright)_{\nu} \subset \mleft( x_n \mright)_n: \exists x \in X :\forall l \in X' : l \mleft( x_{n_{\nu}} \mright) \xrightarrow{\nu \rightarrow \infty} l(x)
	\]
	
One says that $x_{n_{\nu}}\rightharpoonup x$ converges weakly.
\end{lemma}

\begin{remark}
	\begin{enumerate}
		\item[1)] For $p \in (1,\infty)$, $L^{p'} = L^q$, thus $L^{p''} = L^p$, hence for every $\phi \in \L^p$ and $\psi \in L^q$ we must have 
		\[
			\ell_\phi \mleft( \psi \mright)= \int\phi(x) \psi(x) \d x		
		\]
		\item[2)] For a Hilbert space $\mathcal H' = \mathcal H$.
	\end{enumerate}
\end{remark}

\begin{proof}[Proof of \autoref{MinEx}]
	$\mathcal E_{\text{TF}}$ is coercive in the $\|\cdot\|_{\nicefrac{5}{3}}$
-norm and in the $\|\cdot\|_C:= \sqrt{D[\cdot]}$-norm, i.e.
\[
	\mathcal E_{\text{TF}} [p_n] \nearrow \infty \text{ if } \|p_n \|_{\nicefrac{5}{2}} \rightarrow \infty \text{ or if } \|p_n\|_c \rightarrow \infty 
\]

	This is immediate from our lower bound, 
	\[
		\mathcal E_{\text{TF}} (\rho) \geqslant \frac{3}{5} \gamma_{\text{TF}} \|\rho\|_{\nicefrac{5}{3}}^{\nicefrac{5}{3}} - \text{const} \|\rho\|_{\nicefrac{5}{3}} - \text{const} \|\rho\|_{C} + \|\rho\|^2_C,
	\]
	Thus any minimizing sequence $\rho_n$, i.e. $\lim \mathcal E_{\text{TF}} \mleft( \rho_n \mright) = \inf \mathcal E_{\text{TF}} \mleft( \mathcal D \mright)$ must be bounded in $\|\cdot\|_{\nicefrac{5}{3}}$-norm and $\|\cdot\|_C$-norm.


\end{document}
