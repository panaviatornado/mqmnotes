\documentclass[12pt]{article}
\usepackage{titling}
\usepackage{amssymb}
\usepackage{amsmath}
\usepackage{extpfeil}
\usepackage{marvosym}
\usepackage{graphicx}
\usepackage{setspace}
\usepackage{tabto}
\usepackage{wasysym}
\usepackage{textcomp}
\usepackage{braket}
\usepackage{amsthm}
\usepackage{thmtools}
\usepackage[xcolor = RGB, framemethod = TikZ]{mdframed}
%\usepackage[dvipsnames]{xcolor}
\usepackage{subfigure}
\usepackage{mathtools}
\usepackage{ragged2e}
%\usepackage{hhline}
\usepackage{float}
%\usepackage{relsize}
\usepackage{mathrsfs}
\usepackage[margin=1in]{geometry}
\usepackage{units}
%\usepackage{bbm}
%\usepackage{MnSymbol}
\usepackage{varwidth}
\usepackage{titlesec}
%\usepackage[round,colon,authoryear]{natbib}
%\usepackage{nth}
%\usepackage{hyperref}
%\usepackage[russian]{babel}
%\usepackage[utf8]{inputenc}
%\usepackage[T2A,T1]{fontenc}
%\usepackage[latin, british]{babel}

\usepackage{hyperref}
\usepackage{mleftright}
\usepackage{comment}
\usepackage{tensor}
\usepackage{siunitx}


\usepackage{bm}


\numberwithin{equation}{section}

%\titleformat{\subsection}[runin]{\normalfont\large\bfseries}{\thesubsubsection}{1em}{}

\setlength\parindent{0pt}

%\theoremstyle{definition}
%\newtheorem{prop}{Proposition}
%\newtheorem{cor}{Corollary}
%\newtheorem{lemma}{Lemma}


%%%%%%%%%%%%
%%%Colors%%%
%%%%%%%%%%%%

\colorlet{thmline}{red}
\definecolor{thmbg}{rgb}{1,0.9,0.9}
\definecolor{propline}{rgb}{1,0.27,0}
\definecolor{propbg}{rgb}{1,0.854,0.752}
\definecolor{corline}{rgb}{0.812,0.325,0}
\definecolor{corbg}{rgb}{1,0.854,0.752}
\colorlet{lemline}{orange}
\definecolor{lembg}{rgb}{1,0.824,0.702}
\definecolor{exampleline}{rgb}{0,0.898,0.565}
\colorlet{examplebg}{white}
\colorlet{defnline}{blue}
\definecolor{defnbg}{rgb}{0.9,0.9,1}
\definecolor{remline}{rgb}{1,0.843,0}
\colorlet{rembg}{white}

%%%%%%%%%%%%%%%%%%%%
%%%Theorem Styles%%%
%%%%%%%%%%%%%%%%%%%%



\mdfdefinestyle{theoremstyle}{
	middlelinewidth = 3,
	roundcorner  = 10 pt,
	leftmargin = -12,
	rightmargin = -12,
	backgroundcolor  = thmbg,
	middlelinecolor = thmline,
	innertopmargin = -2pt ,
	splittopskip = \topskip ,
	ntheorem = true,
	}

\mdfdefinestyle{propstyle}{
	middlelinewidth = 3,
	roundcorner  = 10 pt,
	leftmargin = -12,
	rightmargin = -12,
	backgroundcolor  = propbg,
	middlelinecolor = propline,
	innertopmargin = -2pt,
	splittopskip = \topskip ,
	ntheorem = true,
	}

\mdfdefinestyle{corstyle}{
	middlelinewidth = 3,
	roundcorner  = 10 pt,
	leftmargin = -12,
	rightmargin = -12,
	backgroundcolor  = corbg,
	middlelinecolor = corline,
	innertopmargin = -2pt ,
	%splittopskip = \topskip ,
	ntheorem = true,
	}

\mdfdefinestyle{lemstyle}{
	middlelinewidth = 3,
	roundcorner  = 10 pt,
	leftmargin = -12,
	rightmargin = -12,
	backgroundcolor  = lembg,
	middlelinecolor = lemline,
	innertopmargin = -2pt ,
	splittopskip = \topskip ,
	ntheorem = true,
	}

\mdfdefinestyle{examplestyle}{
	middlelinewidth = 3,
	roundcorner  = 10 pt,
	leftmargin = -12,
	rightmargin = -12,
	backgroundcolor  = examplebg,
	middlelinecolor = exampleline,
	%innertopmargin = 0pt ,
	splittopskip = \topskip ,
	ntheorem = true,
	}

\mdfdefinestyle{defnstyle}{
	middlelinewidth = 3,
	roundcorner  = 10 pt,
	leftmargin = -12,
	rightmargin = -12,
	backgroundcolor  = defnbg,
	middlelinecolor = defnline,
	%innertopmargin = 0pt ,
	splittopskip = \topskip ,
	ntheorem = true,
	}

\mdfdefinestyle{remstyle}{
	middlelinewidth = 3,
	roundcorner  = 10 pt,
	leftmargin = -12,
	rightmargin = -12,
	backgroundcolor  = rembg,
	middlelinecolor = remline,
	%innertopmargin = 0pt ,
	splittopskip = \topskip ,
	ntheorem = true,
	}


%%%%%%%%%%%%%%%%%%%%%%%%%%
%%%Theorem Environments%%%
%%%%%%%%%%%%%%%%%%%%%%%%%%




\theoremstyle{plain}
\newtheorem{thnum}{Theorem}[section]



\declaretheoremstyle[bodyfont = \normalfont,
]{example}




\declaretheorem[sharenumber = thnum, qed = $\square$, 
preheadhook = {\begin{mdframed}[style = theoremstyle]},
postfoothook = \end{mdframed}
]{theorem}


\declaretheorem[sharenumber = thnum, name = Proposition, qed = $\square$,
preheadhook = {\begin{mdframed}[style = propstyle]},
postfoothook = \end{mdframed}
]{prop}

\declaretheorem[sharenumber = thnum, name = Corollary, qed = $\square$,
preheadhook = {\begin{mdframed}[style = corstyle]},
postfoothook = \end{mdframed}
]{cor}

\declaretheorem[sharenumber = thnum, name = Lemma, qed = $\square$,
preheadhook = {\begin{mdframed}[style = lemstyle]},
postfoothook = \end{mdframed}
]{lemma}

\declaretheorem[style = example, sharenumber = thnum,
preheadhook = {\begin{mdframed}[style = examplestyle]},
postfoothook = \end{mdframed}
]{example}

\declaretheorem[style = example,sharenumber = thnum, name = Definition, qed = $\square$,
preheadhook = {\begin{mdframed}[style = defnstyle]},
postfoothook = \end{mdframed}
]{defn}

\declaretheorem[style = example, sharenumber = thnum, qed = $\square$,
preheadhook = {\begin{mdframed}[style = remstyle]},
postfoothook = \end{mdframed}
]{remark}


%%%%%%%%%%%%%%%%%%%%%%%%%%%%%%%%%%%%%%%%%%%%%%%%%%%%%%%%%%%%%

\declaretheorem[numbered = no, qed = $\square$, name = Theorem,
preheadhook = {\begin{mdframed}[style = theoremstyle]},
postfoothook = \end{mdframed}
]{theorem*}


\declaretheorem[numbered = no, name = Proposition, qed = $\square$,
preheadhook = {\begin{mdframed}[style = propstyle]},
postfoothook = \end{mdframed}
]{prop*}

\declaretheorem[numbered = no, name = Corollary, qed = $\square$,
preheadhook = {\begin{mdframed}[style = corstyle]},
postfoothook = \end{mdframed}
]{cor*}

\declaretheorem[numbered = no, name = Lemma, qed = $\square$,
preheadhook = {\begin{mdframed}[style = lemstyle]},
postfoothook = \end{mdframed}
]{lemma*}

\declaretheorem[style = example, numbered = no, name = Example,
preheadhook = {\begin{mdframed}[style = examplestyle]},
postfoothook = \end{mdframed}
]{example*}

\declaretheorem[style = example, numbered = no, name = Definition, qed = $\square$, name = Definition,
preheadhook = {\begin{mdframed}[style = defnstyle]},
postfoothook = \end{mdframed}
]{defn*}

\declaretheorem[style = example, numbered = no, qed = $\square$, name = Remark,
preheadhook = {\begin{mdframed}[style = remstyle]},
postfoothook = \end{mdframed}
]{remark*}


%Weak Convergence Arrow
\newextarrow{\xrightharpoon}{0529}{\relbar\relbar\rightharpoonup}





\begin{comment}


\theoremstyle{plain}
\newtheorem{thnum}{Theorem}[section]

\declaretheoremstyle[bodyfont = \normalfont,
%postheadspace=0.5em,
%mdframed={
%		backgroundcolor=white, hidealllines=true,
%		skipabove=0pt, skipbelow=0pt,
%		innertopmargin=2pt, innerbottommargin=2pt, innerleftmargin=0pt, innerrightmargin=0pt,
%		}
]{example}
\declaretheorem[sharenumber = thnum, qed = $\square$,shaded={rulecolor=Red,
rulewidth=2pt, bgcolor={rgb}{1,0.9,0.9}, margin = 10pt}]{theorem}
\declaretheorem[sharenumber = thnum, name = Proposition, qed = $\square$,shaded={rulecolor=RedOrange,
rulewidth=2pt, bgcolor={rgb}{1,0.854,0.752}, margin = 10pt}]{prop}
\declaretheorem[sharenumber = thnum, name = Corollary, qed = $\square$,shaded={rulecolor=BurntOrange,
rulewidth=2pt, bgcolor={rgb}{1,0.854,0.752}, margin = 10pt}]{cor}
\declaretheorem[sharenumber = thnum, name = Lemma, qed = $\square$,shaded={rulecolor=Orange,
rulewidth=2pt, bgcolor={rgb}{1,0.824,0.702}, margin = 10pt}]{lemma}
\declaretheorem[style = example, sharenumber = thnum,shaded={rulecolor=SpringGreen,
rulewidth=2pt, bgcolor={rgb}{1,1,1}, margin = 10pt}]{example}
\declaretheorem[style = example,sharenumber = thnum, name = Definition, qed = $\square$,shaded={rulecolor=Blue,
rulewidth=2pt, bgcolor={rgb}{0.9,0.9,1}, margin = 10pt}]{defn}
\declaretheorem[style = example, sharenumber = thnum, qed = $\square$,shaded={rulecolor=Goldenrod,
rulewidth=2pt, bgcolor={rgb}{1,1,1}, margin = 10pt}]{remark}

\declaretheorem[numbered = no, qed = $\square$, name = Theorem]{theorem*}
\declaretheorem[numbered = no, name = Proposition, qed = $\square$,shaded={rulecolor=RedOrange,
rulewidth=2pt, bgcolor={rgb}{1,0.854,0.752}, margin = 10pt}]{prop*}
\declaretheorem[numbered = no, name = Corollary, qed = $\square$,shaded={rulecolor=BurntOrange,
rulewidth=2pt, bgcolor={rgb}{1,0.854,0.752}, margin = 10pt}]{cor*}
\declaretheorem[numbered = no, name = Lemma, qed = $\square$,shaded={rulecolor=Orange,
rulewidth=2pt, bgcolor={rgb}{1,0.824,0.702}, margin = 10pt}]{lemma*}
\declaretheoremstyle[numbered = no, bodyfont = \normalfont]{example*}
\declaretheorem[style = example*, name = Example,qed = $\square$,shaded={rulecolor=lime,
rulewidth=2pt, bgcolor={rgb}{1,1,1}, margin = 10pt}]{example*}
\declaretheorem[style = example*, name = Definition, qed = $\square$,shaded={rulecolor=Blue,
rulewidth=2pt, bgcolor={rgb}{0.9,0.9,1}, margin = 10pt}]{defn*}
\declaretheorem[style = example*, name = Remark, qed = $\square$,shaded={rulecolor=Goldenrod,
rulewidth=2pt, bgcolor={rgb}{1,1,1}, margin = 10pt}]{remark*}

\end{comment}




\newcommand{\nin}{\notin}

\renewcommand{\baselinestretch}{1.3}

\renewcommand{\d}{\mathrm{d}\hspace{-0.025cm}}
\renewcommand\qedsymbol{\textit{q.e.d.}}

\renewcommand{\epsilon}{\varepsilon}
\renewcommand{\phi}{\varphi}
\renewcommand{\theta}{\vartheta}
\renewcommand{\cong}{\simeq}

\DeclareMathOperator*{\esssupp}{ess \, supp}
\DeclareMathOperator*{\supp}{supp}
\DeclareMathOperator{\intr}{int}
\DeclareMathOperator*{\esssup}{ess \, sup}
\DeclareMathOperator{\sgn}{sgn}
\DeclareMathOperator{\rnk}{rnk}
\DeclareMathOperator{\loc}{loc}
\DeclareMathOperator{\vol}{vol}
\DeclareMathOperator{\img}{img}
\DeclareMathOperator{\ran}{ran}
\DeclareMathOperator{\euc}{Eucl}
\DeclareMathOperator{\spn}{span}
\DeclareMathOperator{\id}{Id}
\DeclareMathOperator{\dist}{dist}
\DeclareMathOperator{\Tr}{Tr}
\DeclareMathOperator{\pr}{pr}
\DeclareMathOperator{\spec}{spec}



\let\Re\relax
\DeclareMathOperator{\Re}{\frak R}

\newcommand{\bitext}[1]{\textit{\textbf{#1}}}

\newcommand{\subtitle}[1]{%
  \posttitle{%
    \par\end{center}
    \begin{center}\large#1\end{center}
    \vskip0.5em}%
}

%\renewcommand{\proofname}{\textit{Proof.}}


\makeindex

\begin{document}
\title{Mathematical Quantum Mechanics\\
Prof. Heinz Siedentop}
\author{Martin Peev}
\subtitle{Unofficial Lecture Notes}
\date{}
\maketitle

\newpage

\tableofcontents

\newpage

\setcounter{section}{-1}

\section{Review of Quantum Mechanics}

A state is a non-zero $\psi \in \mathcal H$ where $\mathcal H$ is a Hilbert space. Actually it is an equivalence class with 
\[
	\psi_1 \sim \psi_2 \iff \exists c \in \mathbb{C}^*: c \psi_1 = \psi_2,
\]
so actually it is a ray in $\mathcal H$.

Further the Hilbert space is a $\mathbb{C}$-vector space with a positive definite hermitean inner product $\Braket{\cdot, \cdot}: \mathcal H^2 \rightarrow \mathbb{C}$, that is linear in the second slot, which induces a norm $\|\psi\| = \sqrt{\Braket{\psi,\psi}}$, which makes $\mathcal H$ into a complete normed space.

At least as important are operators or linear maps $A: \mathcal H \rightarrow \mathcal H$.



Stern-Gerlach experiment: The Hamiltonian is given by 
\[
	H_{\text{int}} = \bm S \cdot \bm B(x)
\]
$B$ is mostly in $z$-direction hence we get
\[
	\bm B \cdot \bm S \approx \bm B(x) S_z, 
\]
with
\[
	S_z = \frac{1}{2} \begin{pmatrix}
	1 & 0\\
	0 & -1
	\end{pmatrix}
\]
with the force being proportional to the gradient of the potential, i.e.
\[
	F \sim \bm S \cdot \nabla B,
\]
hence this experiment discriminates the two eigenspaces of $H_{\text{int}}$.

The two roles of the Hamiltonian are thus
\begin{itemize}
	\item giving the time evolution,
	\item measuring (splitting the eigenspaces).
\end{itemize}

$S_z$ is a $2\times 2$ complex matrix, operator on the Hilbert space $\mathbb{C}^2$ everything is simple
\begin{itemize}
	\item it is hermitean $S_z^\dagger = S_z$
	\item discrete eigenvalues, solve $S_z \psi = s_z \psi$, with $s_z \in \mathbb{C}$.
	\item $S_z = \frac{1}{2} \ket{\uparrow}\bra{\uparrow} + \mleft(- \frac{1}{2} \mright) \ket{\downarrow} \bra{\downarrow}$ (weighted sum of eigenprojectors)
	\item defined on all of $\mathbb{C}^2$
	\item norm: maximal |eigenvalue|.
\end{itemize}

(Mathematical) problems arise when $\dim \mathcal H = \infty$. In that situation the same points (but in different order )


\underline{By example:}
\begin{enumerate}
	\item[1)] Eigenvalue equation $A\psi = a \psi$ can have no non-trivial solution in $\mathcal H$, e.g.
	\[
		- \Delta \psi = E \psi \qquad \text{in  } \mathcal H = L^2 \mleft( \mathbb{R}^n \mright).	
	\]
	is only solved by $\psi = 0$ in $\mathcal H$.
	
	"Plane wave solutions":
	\[
		\psi(x) = e^{i k x}	
	\]
	with $\|k\|^2 = E$ are not square integrable, this problem is associated with the continuous spectrum.
	
	Way out: Instead of $A \psi = \lambda \psi$ write $(A - \lambda \id)\psi = 0 $
	or $A-\lambda \id $ is not invertible, with the last assertion being the most general. Hence we shall define
	\[
		\lambda \in \spec(A) :\iff A- \lambda \id \text{ is not invertible}.
	\]
	
	\item[2)] The norm of $A$ is defined as 
	\[
		\|A\|:= \sup_{\psi \in \mathcal H \setminus \{0\}}	\frac{\|A\psi\|}{\|\psi\|} = \sup_{\|\psi\| = 1} = \| A \psi\| \in [0,\infty]
	\]
	If $\|A\| = \infty$, $A$ is called \underline{unbounded}. These cause headaches
	\begin{enumerate}
	
	
	\item[a)] $A$ is not continuous in this case, since
	\[
		A \text{ continuous} \iff \|A\|< \infty	
	\]
	
	Fact of life: $[q,p] = i \implies$ at least one of $p,q$ is unbounded.
	
	\item[b)] In fact $A$ cannot be defined on all of $\mathcal H$, e.g. for $\mathcal H = L^2(\mathbb{R})$, the position operator is given by 
	\[
		(x\psi)(x) = x\psi(x)	
	\]
	can leave $\mathcal H$. For example $\psi(x) = \mleft(1+|x| \mright)^{\nicefrac{2}{3}} \in \mathcal H$, but $x \psi(x) = x \mleft(1+|x| \mright)^{\nicefrac{2}{3}} \notin \mathcal H$.
	
	Hence we have to worry about the domain of definition $\mathcal D(A)$ of an unbounded operator $A$.
	Usually, we want $\mathcal D(A)$ to be dense in $\mathcal H$.
	
	
	\end{enumerate}
	
	\item[3)] The adjoint operator: A nice poperty of Hilbert spaces is that via the scalar product, they are their own duals, i.e. every linear map $f: \mathcal H \rightarrow \mathbb{C}$ defines a $\phi \in \mathcal H$ such that 
	\[
		\forall \psi \in \mathcal H : f(\psi) = \Braket{\phi,\psi}	
	\]
	
	Now take $\phi \in \mathcal H$, such that 
	\[
		f_{\phi}: \begin{aligned} \mathcal D(A) &\longrightarrow \mathbb{C}	\\
			\psi &\longmapsto \Braket{\phi,A\psi}
		\end{aligned}
	\]
	extends to all $\mathcal H$ or equivalently is bounded.
	
	Then there is $\eta_{\phi} \in \mathcal H$ such that $f_{\phi}(\psi) = \Braket{\eta_{\phi},\psi}$.
	
	$\eta_{\phi}$ is usually denoted as ``$A^{\dagger}\phi$". Then 
	\[
		A^{\dagger} \phi \longmapsto \eta_{\phi} = A^{\dagger}\phi	
	\]
	is linear and \[
	\mathcal D\mleft( A^{\dagger} \mright) := \mleft\{ \phi \in \mathcal H \, \big| \, f_{\phi} \text{ is bounded} \mright\}
	\]
	(i.e. you are not free to choose!)
	
	In Quantum mechanics on learns that \underline{self-adjoint} operators have a real spectrum and are potential observables. However, now we have to worry whether $\mathcal D(A) \overset{?}{=} \mathcal D\mleft(A^{\dagger}\mright)$.
	
\end{enumerate}

\begin{defn}
	An operator $A$ is called \underline{symmetric} iff 
	\[
		\forall \phi, \psi \in \mathcal D(A): \; \Braket{\phi, A \psi} = \Braket{A\phi,\psi}	
	\]
	Note that this only says that $A^{\dagger}\big|_{\mathcal D(A)} = A$ (that $\mathcal D(A) \subset \mathcal D \mleft( A^{\dagger} \mright)$ is already implied by the definition). However, it could be that $\mathcal D \mleft( A^{\dagger} \mright) \supsetneq \mathcal D \mleft( A \mright)$.
	
	A symmetric operator $A$ is called \underline{self-adjoint} iff 
	\[
		\mathcal D\mleft(A^{\dagger}\mright) = \mathcal D(A).	
	\]	
	
\end{defn}

\begin{example*}
	Let $\mathcal H = L^2 \big( [0,1]\big)$ and consider the operator $(p\psi)(x) = i \overline{\psi'(x)}$ atleast for \[
	\mathcal D(p) = \mathscr C_a := \mleft\{ f \in \mathcal H \, \big| \, f \text{ absolutely continuous with derivative in } \mathcal H \mright\}.	
	\] Is this operator symmetric?	
	\[
		\Braket{\phi,p\psi}=\int\limits_0^1 \bar\phi(x) i \frac{d}{d x} \phi(x) \d x = \underbrace{ \int\limits_0^1 \overline{i \frac{d}{d x} \phi(x)} \phi(x) \d x	}_{ = \Braket{p\phi,\psi}} + i \bar \phi(x) \psi(x) \big|_{0}^1
	\]
	Since the last term has to equal 0, for $p$ to be symmetric, we have to impose the condition $\psi(0) = \psi(1) = 0$ so actually 
	\[
		\mathcal D(p) = \mleft\{ \psi \in \mathscr C_{a} \, \big| \, \psi|_{\{0,1\}} = 0 \mright\} 	
	\]
	Imposing these boundary conditions on $\psi$ makes it unnecessary for $\phi$, hence $\mathcal D\mleft(A^{\dagger}\mright) \supsetneq \mathcal D(A)$.
	
	One eigenvalue of the operator $p^{\dagger}$ is $\pm i$, since
	\[
		i \frac{d}{d x} \phi = \pm i \phi	\implies \phi = e^{\pm i x} \in \mathscr C_a.
	\]
	In particular we get that $\mathcal D \mleft(p^*\mright) = \mathscr C_a$.
	
	However, this means that $p^* \psi_{\pm} = \pm i \psi_{\pm}$	which means that $\Braket{\psi,p^*\psi}$ is not necessarily real, i.e. $p^*$ is not an observable. 
	
	We observe that $\dim \mathcal{D}(p^*)/\mathcal{D}(p) = 2$.
	
	$\mathcal D_{\pm} = \mleft\{ \text{eigenspace of eigenvalue }\pm i \mright\}$
	
\end{example*}

To solve this problem a possible strategy is to enlarge $\mathcal D(p)$ by a one-dimensional subspace of $\mathcal{D}(p^*)$, $\mathcal D(p_e)$ such that $\mathcal D(p_e^*) = \mathcal D (p_e)$.

We have to find to find a good one dimensional subspace $\mathcal D_a$ of $\mathcal D_+ \oplus \mathcal{D}_-$. Hence let 
\[
	\psi \in \mathcal D_a \; \qquad \psi = \tilde \psi_+ + \tilde \psi_i, \quad \tilde\psi_{\pm} \in \mathcal D_{\pm}.
\]
This decomposition is unique and defines a map $S: \mathcal D_+ \rightarrow \mathcal{D}_-$ such that
\[
	\tilde\psi_+ \in\mathcal{D}_+ \implies \tilde\psi_+ + S\mleft( \tilde\psi_+ \mright) \in \mathcal D_a.
\]

To see whether the new extended $p_e$ is symmetric for this
\[
	0 = \Braket{p^* \mleft(\tilde\psi_+ + S \tilde\psi_+  \mright) , \tilde\psi_+ + S\tilde\psi_+ } - \Braket{ \tilde\psi_+ + S\tilde\psi_+ , p^* \tilde\psi_+ + S\tilde\psi_+} = \cdots = -2 \mleft( \Braket{\tilde\psi_+,\tilde\psi_+} - \Braket{S\tilde\psi_+,S\tilde\psi_+} \mright),
\]
	where $p^* \psi_+ = i \psi_+$.
	
	Hence for $p_e$ to be symmetric we need $S$ to be unitary. It turns out that this condition is also sufficient.


In the general case we also have 
\[
	\mathcal D \mleft( A \mright) \oplus \mathcal D_+ \oplus \mathcal D_- = \mathcal{D} \mleft(A^+ \mright).
\]
We have the freedom to choose a self-adjoint extension for each $S: \mathcal D_+ \rightarrow \mathcal D_-$ unitary. Then 
\[
	\mathcal D_a = \mleft\{ \psi_+ + S \psi_+ \, \big| \, \psi \in \mathcal D_+ \mright)
\]

In our case, $\dim \mathcal D_{\pm} = $, so the possible unitary operators are $S_{\alpha}\psi_+ = e^{i \alpha} \psi_-$, for $\alpha \in [0,2\pi)$. Thus for $\psi \in \mathcal D(q) + \mathcal D_a$ we have
\[
	\psi(0) = e^{i \beta} \psi(1).
\]
I.e. $p$ is self-adjoint on the quasi-periodic functions, and thus the spectrum of an extension $p_{\alpha}$, for a given choice of $\alpha$, is $2\pi \mathbb Z + \alpha$.

\begin{theorem*}[Stone's Theorem]
		For every self-adjoint operator $A$ there exists a unitary representation $U: \mathbb{R} \rightarrow U (\mathcal H), \; U (t) U(s) = U(t+s) $ such that 
		\[
			\forall \psi \in \mathcal D(A): \frac{d}{d t} U(t) \psi \Big|_{t=0} = i A \psi		
		\]
\end{theorem*}

\newpage


The standard quantum-mechanical ``self-adjoint" operator is $H_0 = - \Delta$. For $\psi \in \mathscr C_0^{\infty} \mleft( \mathbb{R}^3 \mright)$ we have
\[
	\Braket{\phi,-\Delta \psi} = \sum_{j} \int\limits_{\mathbb{R}^3} \bar\phi(x) \mleft( - \Delta \psi \mright)(x) \d x = \sum_j \int \overline{ \nabla \phi}(x) \cdot \nabla \psi(x) \d x = \sum_j \int \overline{- \Delta \phi}(x)\psi(x) = \Braket{-\Delta \phi , \psi}
\]

Since $\mathscr C_0^{\infty} \mleft( \mathbb{R}^3 \mright)$ is dense in $L^2 \mleft( \mathbb{R}^3 \mright)$ we can extend this operator to all $L^2\mleft( \mathbb{R}^3 \mright)$.

Other examples of operators are 
\begin{align*}
	H_z &= - \Delta - \frac{z}{\|x\|}\\
	H & = - \Delta + x^2\\
	H& = - \Delta - \alpha \chi_M
\end{align*}

\begin{defn}[Fourier Transform]
	The Fourier transform of $\psi \in L^1\mleft( \mathbb R^n \mright) $ is given by 
	\[
		\mleft(\mathcal F \psi \mright)(\xi) = \frac{1}{\mleft(2\pi \mright)^{\nicefrac{n}{2}}} \int\limits_{\mathbb{R}^n} e^{- i \xi \cdot x} \psi(x) \d x 
	\]
\end{defn}

\begin{defn}[Japanese Symbol]
	\[
		\Braket{x} = \sqrt{1+\|x\|^2}	
	\]
\end{defn}

For $\psi(x):= \frac{1}{\Braket{x}^{3-\alpha}}$ it holds that $\psi \notin L^1$ but $\psi \in L^2$, for $\alpha$ small.

\begin{defn}
	Let us consider the semi-norm
	\[
		\|\psi\|_{\alpha, \beta} := \sup_{x \in \mathbb{R}^n} \mleft| x^\alpha \mleft( \partial^\beta \psi \mright)(x) \mright|,	
	\]
	with $\alpha, \beta \in \mathbb{N}_0^{n}$, where 
	\[
		x^\alpha= x^{\alpha_1}_1 \cdots x^{\alpha_n}_n, \quad \partial^\beta := \partial^{\alpha_1}_1	\cdots \partial^{\alpha_n}_n.
	\]
	Then the Schwartz space on $\mathbb{R}^n$ is defined as
	\[
		\mathscr S \mleft( \mathbb{R}^n \mright) := \mleft\{ \psi \in \mathbb{C}^{\infty} \mleft( \mathbb{R}^n \mright) \, \big| \, \|\psi\|_{\alpha, \beta} < \infty \mright\},
	\]
	with the complete metric 
	\[
		d(\psi,\phi) := \sum_{\alpha, \beta \in \mathbb{N}^n} 2^{-|\alpha|-|\beta|}	\frac{\|\psi-\phi\|_{\alpha,\beta}}{1+\|\psi-\phi\|_{\alpha,\beta}}.
	\]
\end{defn}
\begin{defn}[Tempered Distributions]
	The dual space of $\mathscr{S}$, 
	\[
		\mathscr{S}' := \mleft\{ T : \mathscr{S}\mleft(\mathbb{R}^n \mright) \rightarrow \mathbb{C} \, \big| \, \text{linear, continuous} \mright\}	
	\]
	is called the space of tempered distributions.
\end{defn}

We shall define the Fourier transform on $\mathcal S'$ by 
\[
	\Braket{\mathcal FL , \psi} := \Braket{L, \mathcal F \psi},
\]
where $L \in \mathscr S', \psi \in \mathscr S$ and $\Braket{L,\psi}:= L(\psi)$.

The delta function $\delta_x : f \mapsto f(x)$ is a continuous functional on $\mathcal S$ since
\[
	\forall \psi \in S: \mleft| \delta_x f \mright| = |f(x)| \leqslant \|f\|_{0,0}.
\]
The Fourier transform of the delta function is given by
\[
	\Braket{\mathcal F \delta , \phi} = \Braket{\delta, \hat \phi} = \hat \phi(0) = \frac{1}{\mleft(2\pi \mright)^{\nicefrac{n}{2}}} \int e^{i 0 x} \phi(x) \d x = \Braket{\frac{1}{(2\pi)^{\nicefrac{n}{2}}}, \phi},
\]
i.e. $\mathcal F \delta = \frac{1}{(2\pi)^{\nicefrac{n}{2}}}$.


If $g$ is measurable and 
\[
	\exists C \exists M \forall x \in \mathbb{R}^n \mleft| \Braket{x}^{-M} g(x) \mright| < C
\]
then let $L_g$ be the functional defined by
\[
	\Braket{L_g, \phi}:= \int g(x) \phi(x) \d x
\]
for all $\phi \in \mathscr S\mleft(\mathbb{R}^n \mright)$.
One has to check that 
\begin{enumerate}
	\item[1)] $L_g$ is a continuous functional.
	\item[2)] $L_g$ with $g \in L^2 \mleft( \mathbb{R}^n \mright)$ is a tempered distribution.
\end{enumerate}


\begin{defn}[Sobolev space]
	A Sobolev space 
	\[
		H^{\alpha} \mleft( \mathbb{R}^d \mright) = \mleft\{	\psi \, \big| \, \mleft\| \Braket{\xi}^{\alpha} \hat\psi \mright\|_{L^2} < \infty \mright\}.
	\]
	Further we define 
	\begin{align*}
		H^1 \mleft( \mathbb{R}^n \mright): \Braket{\phi,\psi}_{H_1} := \int \Braket{\xi}^2 \overline{\hat\phi}(\xi) \overline{\psi}(\xi) \d \xi \\
		H^2 \mleft( \mathbb{R}^n \mright): \Braket{\phi,\psi}_{H_2} := \int \Braket{\xi}^4 \overline{\hat\phi}(\xi) \hat\psi(\xi) \d \xi
	\end{align*}
\end{defn}

On this space we can define the Hamiltonian 
\[
	-\Delta : \begin{aligned}
		H^2 \mleft( \mathbb{R}^n \mright) \subset L^2 \mleft( \mathbb{R}^n \mright) &\longrightarrow L^2 \mleft(\mathbb{R}^n \mright)\\
		\psi & \longmapsto \mathcal F^{-1} \mleft( |\cdot|^2 \hat\psi \mright)
	\end{aligned}
\]

Check that this is sefl-adjoint under the assumption that $F$ is unitary.
\[
	-\Delta: \mathcal{F}^-1 |\xi|^2 \mathcal F
\]

\end{document}